\documentclass[a4paper]{article} % A4 paper and 11pt font size

\usepackage{braket}
\usepackage{amsmath}
\usepackage{amssymb}
\usepackage{bm}
\usepackage[utf8]{inputenc}
\usepackage{verbatim}
\usepackage{tikz}
%\usepackage{tikz-feynman}
\usepackage{pgfplots}
\usepackage{pgffor}
\usepackage[version-1-compatibility]{siunitx}
\usepackage{fancyhdr}

\usepackage{hyperref}

\usepackage{geometry}

 \geometry{
 a4paper,
 total={210mm,297mm},
 left=28mm,
 right=28mm,
 top=30mm,
 bottom=40mm,
 }


\usepackage{framed}

\usepackage{amssymb} %for Lagrangian L, order O
\usepackage{cancel} %for strikethroughs
\usepackage{slashed} %for Feynman slashes

\newcommand{\me}{\mathrm{e}}
\newcommand{\D}{\mathcal{D}}
\newcommand{\pmx}[1]{\begin{pmatrix}#1\end{pmatrix}}
\newcommand{\m}{\mathcal{M}}
\newcommand{\br}{\mathcal{B}}
\DeclareMathOperator{\im}{Im}
\DeclareMathOperator{\re}{Re}

\newcommand{\bpp}{\(B^0 \to \pi^+\pi^-\)}
\newcommand{\bpk}{\(B^0 \to \phi K_S\)}
\newcommand{\tr}[1]{\text{Tr}\left[#1\right]}
\newcommand{\mev}{\text{ MeV}}
\newcommand{\gev}{\text{ GeV}}

\renewcommand{\c}{\mathcal{C}}
\newcommand{\p}{\mathcal{P}}

\usepackage{gensymb}

%%%%%%%%%%%%%%%CATE'S PREAMBLE BIT%%%%%%%%%%%%%%%%

\usepackage{tikz}
\usetikzlibrary{arrows,shapes}
\usetikzlibrary{trees}
\usetikzlibrary{patterns}
\usetikzlibrary{matrix,arrows} 				% For commutative diagram
											% http://www.felixl.de/commu.pdf
\usetikzlibrary{positioning}				% For "above of=" commands
\usetikzlibrary{calc,through}				% For coordinates
\usetikzlibrary{decorations.pathreplacing}  % For curly braces
% http://www.math.ucla.edu/~getreuer/tikz.html
\usepackage{pgffor}							% For repeating patterns

\usetikzlibrary{decorations.pathmorphing}	% For Feynman Diagrams
\usetikzlibrary{decorations.markings}
\tikzset{
	 >=stealth', %%  Uncomment for more conventional arrows
    vector/.style={decorate, decoration={snake}, draw},
	provector/.style={decorate, decoration={snake,amplitude=2.5pt}, draw},
	antivector/.style={decorate, decoration={snake,amplitude=-2.5pt}, draw},
    fermion/.style={draw=black, postaction={decorate},
        decoration={markings,mark=at position .55 with {\arrow[draw=black]{>}}}},
    fermionbar/.style={draw=black, postaction={decorate},
        decoration={markings,mark=at position .55 with {\arrow[draw=black]{<}}}},
    fermionnoarrow/.style={draw=black},
    gluon/.style={decorate, draw=black,
        decoration={coil,amplitude=4pt, segment length=5pt}},
    scalar/.style={dashed,draw=black, postaction={decorate},
        decoration={markings,mark=at position .55 with {\arrow[draw=black]{>}}}},
    scalarbar/.style={dashed,draw=black, postaction={decorate},
        decoration={markings,mark=at position .55 with {\arrow[draw=black]{<}}}},
    scalarnoarrow/.style={dashed,draw=black},
    electron/.style={draw=black, postaction={decorate},
        decoration={markings,mark=at position .55 with {\arrow[draw=black]{>}}}},
    bigvector/.style={decorate, decoration={snake,amplitude=4pt}, draw},
    arrow/.style={draw=black, postaction={decorate},
        decoration={markings,mark=at position 1 with {\arrow[draw=black]{>}}}},
}

% TIKZ - for block diagrams, 
% from http://www.texample.net/tikz/examples/control-system-principles/
% \usetikzlibrary{shapes,arrows}
\tikzstyle{block} = [draw, rectangle, 
    minimum height=3em, minimum width=6earticlem]

%%%%%%%%%%%%%%END CATE'S PREAMBLE BIT%%%%%%%%%%%%%


\usepackage{fancyhdr}
\usepackage{pdflscape}
\usepackage{bm}

%for side-by-side figures
\usepackage{graphicx}
\usepackage{caption}
\usepackage{subcaption}


%Curly epsilons
\let\oldepsilon\epsilon
\let\epsilon\varepsilon


\setlength{\parindent}{2em}
\setlength{\parskip}{1em}
\renewcommand{\baselinestretch}{1.1}

%----------------------------------------------------------------------------------------
%	TITLE SECTION
%----------------------------------------------------------------------------------------
\setlength\parindent{0pt} % Removes all indentation from paragraphs - comment this line for an assignment with lots of text


\pagenumbering{arabic}
\begin{document}
\pagestyle{empty}

\newcommand{\HRule}{\rule{\linewidth}{0.5mm}}

\begin{titlepage}

    \begin{center}
        \textsc{\large SN: 587623}\\[6cm]

        \HRule \\[0.5cm]
		\Huge \textbf{PHYC90011 Particle Physics}\\[0.5cm]
        \huge \textbf{Assignment 5}\\[0.5cm] 
        \HRule \\[1.5cm]
        \begin{minipage}{0.4\textwidth}
        \begin{center}

        \large By \\[0.75cm]
        \huge Braden \scshape Moore \\[0.5cm]
        \normalsize \normalfont Master of Science \\
        The University of Melbourne \\

        \end{center}
        \end{minipage}

        \vfill

        \large \today
    \end{center}

\newpage
\end{titlepage}
%----------------------------------------------------------------------------------------
\pagestyle{fancy}
\pagenumbering{arabic}
\rfoot{\textsc{Braden Moore, 587623}}
\lfoot{\textsc{\today}}
\rhead{\textsc{Particle Physics, Part 2: Assignment 3}}
\setcounter{page}{1}
\section{Drag racing in space in the Greco-Roman era}
%%%%%%%%%%%%%%%%%%%%QUESTION 1%%%%%%%%%%%%%%%%%%%%%
\begin{framed}
Buoyed by the success of their intrepid interstellar experiment on the twin paradox (the legend is recounted in Section 1.13 of the 1st edition of A First Course in General Relativity, by B. F. Schutz), twin sisters Artemis and Diana decide to conduct a follow-up investigation into the physics of accelerated reference frames in flat space as follows. At the same instant, 1 the sisters jump into two rocket ships, which are initially at rest in a global inertial frame, 2 and race off in the x-direction with constant but unequal proper accelerations. In their space packs they carry a laser pointer (Artemis), a mirror (Diana), and identically manufactured clocks and rulers (Artemis and Diana). 3 Their departure points and trajectories are not the same. In the global inertial frame in which the rocket ships are initially at rest, Artemis starts out from the event $(t, x, y, z) = (0, g^{-1}, 0, 0)$, where $(t, x, y, z)$ are standard Minkowski coordinates, and g denotes her proper acceleration. As derived in lectures, her trajectory on a spacetime diagram is a hyperbola in the x-t plane defined parametrically by
\begin{align}
t_A(\tau A) &= g^{-1} \sinh g\tau_A,\label{eq1}\\
x_A(\tau A) &= g^{-1} \cosh g\tau_A,\label{eq2}\\
y_A(\tau A) &= 0,\\
z_A(\tau A) &= 0
\end{align}

where $\tau_A$ is her proper time. Diana’s departure point, trajectory, and proper acceleration remain unspecified for now.
\end{framed}

\begin{framed}
\textbf{(a)}
Before attempting any experiments, Artemis constructs a new coordinate system $(t, x, y, z)$ for making measurements in the neighbourhood of her rocket ship. The construction proceeds in three steps, which we copy here. 
\end{framed}

\begin{framed}
\textbf{i.} 
Fermi-Walker transport the Minkowski basis vectors $\vec{e}_0$, $\vec{e}_1$, $\vec{e}_2$, and $\vec{e}_3$ along Artemis’s world line. Show that this procedure yields a new basis
\begin{align}
\vec{e}_0(\tau_A) &= \vec{e}_0 \cosh g\tau_A +\vec{e}_1 \sinh g\tau_A,\label{ai. eq1}\\
\vec{e}_1(\tau_A) &= \vec{e}_0 \sinh g\tau_A +\vec{e}_1 \cosh g\tau_A,\\
\vec{e}_2(\tau_A) &= \vec{e}_2,\\
\vec{e}_3(\tau_A) &= \vec{e}_3\label{ai. eq4}
\end{align}

at the point along the world line labelled by $\tau_A$.
\end{framed}

\begin{framed}
\textbf{ii.} 
Verify that one can obtain (\ref{ai. eq1})-(\ref{ai. eq4}) by Lorentz boosting the Minkowski basis vectors into the rocket ship’s momentarily comoving reference frame.
\end{framed}

\begin{framed}
\textbf{iii.} 
Let the vector $x'\vec{e}_{1'}+y'\vec{e}_{2'}+z'\vec{e}_{3'}$ be the displacement of an event $P$ from the spacetime location of the centre of mass of Artemis’s rocket ship. Let $t'$ be the proper time measured by Artemis at the space point $(x,y,z)=(0,0,0)$. Show that these definitions lead to the transformation
\begin{align}
t&=(g^{-1}+x')\sinh gt',\\
x&=(g^{-1}+x') \cosh gt',\\
y&=y'\\
z&=z',
\end{align}
if $P$ occurs at $(t,x,y,z)$ in Minkowski coordinates.
\end{framed}


\begin{framed}
\textbf{(b)} 
The primed coordinates are curvilinear, even though the spacetime is flat.
\end{framed}

\begin{framed}
\textbf{i.} Transform the Minkowski metric of the global inertial frame into the primed coordinates. You should find
\begin{equation}
g_{\alpha' \beta'}=\text{diag}[-(1+gx')^2,1,1,1].
\end{equation}
\end{framed}

\begin{framed}
\textbf{ii.} Is it a worry that we find $g_{0'0'}=-(1+gx')^2$ yet at the same time we compute $\vec{e}_{0'}\cdot\vec{e}_{0'}=-1$ from (\ref{ai. eq1})? Why?
\end{framed}

\begin{framed}
\textbf{iii.} Explain physically why there is a coordinate singularity at $x'=-g^{-1}$.
\end{framed}

\begin{framed}
\textbf{iv.} Sketch the locus of singular events on the $x-t$ plane. Comment on the physical significance of your result.
\end{framed}

\begin{framed}
\textbf{v.} Show that the only nonzero Christoffel symbols in the primed coordinates are
\begin{align}
\Gamma^{0'}_{0'1'}&=g(1+gx')^{-1}=\Gamma^{0'}_{1'0'}\\
\Gamma^{1'}_{0'0'}&=g(1+gx').
\end{align}
\end{framed}

\begin{framed}
\textbf{(c)} Artemis looks out of the windscreen of her rocket ship and “sees” Diana hovering a constant distance away. (We explore what “sees” means further below.) Putting her ruler and clock in her pocket, she opens the airlock, adjusts the mini-thrusters on her space suit, and heads out (slowly, to avoid introducing unwanted relativistic distortions!) to measure the coordinate distance to her sister. Laying the ruler end to end, she confirms that Diana hovers at rest at the spatial location $(x',y',z')=(h,0,0)$ for all $t'$.
\end{framed}

\begin{framed}
\textbf{i.} Calculate Diana’s 4-velocity in the primed coordinates. You should find
\begin{align}
u^{0'}&=(1+gh)^{-1},\label{ci. eq1}\\
u^{1'}&=0.
\end{align}
\end{framed}

\begin{framed}
\textbf{ii.} Calculate Diana’s 4-acceleration in the primed coordinates. You should find
\begin{align}
a^{0'}&=0,\\
a^{1'}&=g(1+gh)^{-1}.\label{cii. eq 2}
\end{align}
\end{framed}

\begin{framed}
\textbf{iii.} Explain in words why equations (\ref{ci. eq1})-(\ref{cii. eq 2}) imply that Diana's \emph{proper acceleration} is constant but different from Artemis’s proper acceleration.
\end{framed}

\begin{framed}
\textbf{iv.} Argue (carefully!) that the proper time $\Delta\tau_D$ between ticks of Diana’s clock on board her rocket ship is related to the proper time $\Delta\tau_A$ between ticks of Artemis’s clock on board her rocket ship according to
\begin{equation}
\frac{\Delta\tau_A}{\Delta\tau_D}=\frac{1}{1+gh}
\end{equation}
\end{framed}

\begin{framed}
\textbf{v.} Combine all of the above results to show that Diana’s trajectory in the global inertial frame is given by
\begin{align}
t_D(\tau_D)&=(g^{-1}+h)\sinh[(g^{-1}+h)^{-1}\tau_D],\label{eq21}\\
x_D(\tau_D)&=(g^{-1}+h)\cosh[(g^{-1}+h)^{-1}\tau_D],\label{eq22}
\end{align}
ith departure point $(t,x,y,z)=(0,g^{-1}+h,0,0)$. Please note that there are many valid ways to approach this part of the question, e.g. by transforming the relevant vectors from primed to Minkowski coordinates.
\end{framed}

\begin{framed}
\textbf{(d)} Artemis and Diana conduct a laser ranging experiment with the supplies in their space packs. Back in her own rocket ship, Artemis fires her laser pointer straight at Diana. Diana holds up her mirror and reflects the light ray back to Artemis. Artemis measures the round-trip time in the primed coordinate system and infers the \emph{radar distance}.
\end{framed}

\begin{framed}
\textbf{i.} Show that the path $[t'(\lambda),x'(\lambda)]$ traced by a photon on its way from Artemis to Diana (where $\lambda$ is an affine parameter) satisfies
\begin{equation}
\frac{dt'}{dx'}=\frac{1}{1+gx}\label{di. eq1}
\end{equation}
\end{framed}

\begin{framed}
\textbf{ii.} From (\ref{di. eq1}), show that the radar distance is given by $g^{-1}\ln (gh+1)$. The radar distance is is always longer than the coordinate distance $h$.
\end{framed}

\begin{framed}
\textbf{iii.} Draw a spacetime diagram in Minkowski coordinates, i.e. in the $x-t$ plane, showing the trajectories of Artemis, Diana, and the photon round trip.
\end{framed}

\begin{framed}
\textbf{iv.} Calculate the Minkowski coordinates of the three key events that define the laser ranging experiment: photon emitted by Artemis, photon reflected by Diana, photon received back by Artemis. You should find
\begin{align}
(t,x)&=(0,g^{-1}),\\
(t,x)&=(2g^{-1})[(1+gh)^2-1,(1+gh)^2+1],\\
(t,x)&=(2g^{-1})[(1+gh)^2-(1+gh)^{-2},(1+gh)^2+(1+gh)^{-2}].
\end{align}
\end{framed}

\begin{framed}
\textbf{(e)} Encouraged by the laser ranging results, the sisters attempt a ``speed camera'' experiment. Before blasting off in their rocket ships, they compare notes while at rest in the global inertial frame and agree that the proper frequency of the laser pointer is $\nu_e$. Once in flight, Artemis fires her laser pointer at Diana, and Diana reflects the light ray back to Artemis. Both sisters look for a Doppler shift in the frequency of the light ray they receive.
\end{framed}

\begin{framed}
\textbf{i.} Write down Artemis's and Diana's 4-velocities in the primed coordinates.
\end{framed}

\begin{framed}
\textbf{ii.} Hence, or otherwise, prove that the frequency $\nu_D$ that Diana measures in the proper frame of her rocket ship is given by
\begin{equation}
\frac{\nu_D}{\nu_e}=\frac{1}{1+gh}\label{eii. eq1}
\end{equation}
\end{framed}

\begin{framed}
\textbf{iii.} Equation (\ref{eii. eq1}) says that Diana observes a nonzero Doppler shift. Yet Artemis and Diana are both at rest in the primed coordinates. Resolve this apparent paradox in physical terms.
\end{framed}

\begin{framed}
\textbf{iv.} \emph{(Optional.)} Prove that the frequency $\nu_A$ of the reflected ray that Artemis measures in the proper frame of her rocket ship equals $\nu_e$, i.e. Artemis does not observe a Doppler shift.
\end{framed}

\begin{framed}
\textbf{v.} \emph{(Optional.)} In the global inertial frame, Diana and her mirror move relative to Artemis, as we can see from equations (\ref{eq1}), (\ref{eq2}), (\ref{eq21}), and (\ref{eq22}). How is this possible, given the result in part (e)(iv)? Justify your answer in words. It is ``fun'' (okay, ``instructive'') to repeat the calculation in Minkowski coordinates to appreciate fully what is going on.
\end{framed}

\begin{framed}
\textbf{vi.} \emph{(Optional.)} Does your result in part (e)(iv) change, if Diana grabs hold of the mirror and moves it with speed V in the x -direction in the proper frame of her rocket ship? You should find that the answer is ``yes'' and specifically that one has
\begin{equation}
\frac{\nu_A}{\nu_e}=\left(1-\frac{V'}{1+gh}\right)\left(1+\frac{V'}{1+gh}\right)^{-1}.
\end{equation}
\end{framed}




\end{document}







