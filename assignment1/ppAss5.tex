\documentclass[a4paper]{article} % A4 paper and 11pt font size

\usepackage{braket}
\usepackage{amsmath}
\usepackage{amssymb}
\usepackage{bm}
\usepackage[utf8]{inputenc}
\usepackage{verbatim}
\usepackage{tikz}
%\usepackage{tikz-feynman}
\usepackage{pgfplots}
\usepackage{pgffor}
\usepackage[version-1-compatibility]{siunitx}
\usepackage{fancyhdr}

\usepackage{hyperref}

\usepackage{geometry}

 \geometry{
 a4paper,
 total={210mm,297mm},
 left=28mm,
 right=28mm,
 top=30mm,
 bottom=40mm,
 }


\usepackage{framed}

\usepackage{amssymb} %for Lagrangian L, order O
\usepackage{cancel} %for strikethroughs
\usepackage{slashed} %for Feynman slashes

\newcommand{\me}{\mathrm{e}}
\newcommand{\D}{\mathcal{D}}
\newcommand{\pmx}[1]{\begin{pmatrix}#1\end{pmatrix}}
\newcommand{\m}{\mathcal{M}}
\newcommand{\br}{\mathcal{B}}
\DeclareMathOperator{\im}{Im}
\DeclareMathOperator{\re}{Re}

\newcommand{\bpp}{\(B^0 \to \pi^+\pi^-\)}
\newcommand{\bpk}{\(B^0 \to \phi K_S\)}
\newcommand{\tr}[1]{\text{Tr}\left[#1\right]}
\newcommand{\mev}{\text{ MeV}}
\newcommand{\gev}{\text{ GeV}}

\renewcommand{\c}{\mathcal{C}}
\newcommand{\p}{\mathcal{P}}

\usepackage{gensymb}

%%%%%%%%%%%%%%%CATE'S PREAMBLE BIT%%%%%%%%%%%%%%%%

\usepackage{tikz}
\usetikzlibrary{arrows,shapes}
\usetikzlibrary{trees}
\usetikzlibrary{patterns}
\usetikzlibrary{matrix,arrows} 				% For commutative diagram
											% http://www.felixl.de/commu.pdf
\usetikzlibrary{positioning}				% For "above of=" commands
\usetikzlibrary{calc,through}				% For coordinates
\usetikzlibrary{decorations.pathreplacing}  % For curly braces
% http://www.math.ucla.edu/~getreuer/tikz.html
\usepackage{pgffor}							% For repeating patterns

\usetikzlibrary{decorations.pathmorphing}	% For Feynman Diagrams
\usetikzlibrary{decorations.markings}
\tikzset{
	 >=stealth', %%  Uncomment for more conventional arrows
    vector/.style={decorate, decoration={snake}, draw},
	provector/.style={decorate, decoration={snake,amplitude=2.5pt}, draw},
	antivector/.style={decorate, decoration={snake,amplitude=-2.5pt}, draw},
    fermion/.style={draw=black, postaction={decorate},
        decoration={markings,mark=at position .55 with {\arrow[draw=black]{>}}}},
    fermionbar/.style={draw=black, postaction={decorate},
        decoration={markings,mark=at position .55 with {\arrow[draw=black]{<}}}},
    fermionnoarrow/.style={draw=black},
    gluon/.style={decorate, draw=black,
        decoration={coil,amplitude=4pt, segment length=5pt}},
    scalar/.style={dashed,draw=black, postaction={decorate},
        decoration={markings,mark=at position .55 with {\arrow[draw=black]{>}}}},
    scalarbar/.style={dashed,draw=black, postaction={decorate},
        decoration={markings,mark=at position .55 with {\arrow[draw=black]{<}}}},
    scalarnoarrow/.style={dashed,draw=black},
    electron/.style={draw=black, postaction={decorate},
        decoration={markings,mark=at position .55 with {\arrow[draw=black]{>}}}},
    bigvector/.style={decorate, decoration={snake,amplitude=4pt}, draw},
    arrow/.style={draw=black, postaction={decorate},
        decoration={markings,mark=at position 1 with {\arrow[draw=black]{>}}}},
}

% TIKZ - for block diagrams, 
% from http://www.texample.net/tikz/examples/control-system-principles/
% \usetikzlibrary{shapes,arrows}
\tikzstyle{block} = [draw, rectangle, 
    minimum height=3em, minimum width=6earticlem]

%%%%%%%%%%%%%%END CATE'S PREAMBLE BIT%%%%%%%%%%%%%


\usepackage{fancyhdr}
\usepackage{pdflscape}
\usepackage{bm}

%for side-by-side figures
\usepackage{graphicx}
\usepackage{caption}
\usepackage{subcaption}


%Curly epsilons
\let\oldepsilon\epsilon
\let\epsilon\varepsilon


\setlength{\parindent}{2em}
\setlength{\parskip}{1em}
\renewcommand{\baselinestretch}{1.1}

%----------------------------------------------------------------------------------------
%	TITLE SECTION
%----------------------------------------------------------------------------------------
\setlength\parindent{0pt} % Removes all indentation from paragraphs - comment this line for an assignment with lots of text


\pagenumbering{arabic}
\begin{document}
\pagestyle{empty}

\newcommand{\HRule}{\rule{\linewidth}{0.5mm}}

\begin{titlepage}

    \begin{center}
        \textsc{\large SN: 587623}\\[6cm]

        \HRule \\[0.5cm]
		\Huge \textbf{PHYC90011 Particle Physics}\\[0.5cm]
        \huge \textbf{Assignment 5}\\[0.5cm] 
        \HRule \\[1.5cm]
        \begin{minipage}{0.4\textwidth}
        \begin{center}

        \large By \\[0.75cm]
        \huge Braden \scshape Moore \\[0.5cm]
        \normalsize \normalfont Master of Science \\
        The University of Melbourne \\

        \end{center}
        \end{minipage}

        \vfill

        \large \today
    \end{center}

\newpage
\end{titlepage}
%----------------------------------------------------------------------------------------
\pagestyle{fancy}
\pagenumbering{arabic}
\rfoot{\textsc{Braden Moore, 587623}}
\lfoot{\textsc{\today}}
\rhead{\textsc{Particle Physics, Part 2: Assignment 3}}
\setcounter{page}{1}
\section{Drag racing in space in the Greco-Roman era}
%%%%%%%%%%%%%%%%%%%%QUESTION 1%%%%%%%%%%%%%%%%%%%%%
\begin{framed}
Buoyed by the success of their intrepid interstellar experiment on the twin paradox (the legend is recounted in Section 1.13 of the 1st edition of A First Course in General Relativity, by B. F. Schutz), twin sisters Artemis and Diana decide to conduct a follow-up investigation into the physics of accelerated reference frames in flat space as follows. At the same instant, 1 the sisters jump into two rocket ships, which are initially at rest in a global inertial frame, 2 and race off in the x-direction with constant but unequal proper accelerations. In their space packs they carry a laser pointer (Artemis), a mirror (Diana), and identically manufactured clocks and rulers (Artemis and Diana). 3 Their departure points and trajectories are not the same. In the global inertial frame in which the rocket ships are initially at rest, Artemis starts out from the event $(t, x, y, z) = (0, g^{-1}, 0, 0)$, where $(t, x, y, z)$ are standard Minkowski coordinates, and g denotes her proper acceleration. As derived in lectures, her trajectory on a spacetime diagram is a hyperbola in the x-t plane defined parametrically by
\begin{align}
t_A(\tau A) &= g^{-1} \sinh g\tau_A,\\
x_A(\tau A) &= g^{-1} \cosh g\tau_A,\\
y_A(\tau A) &= 0,\\
z_A(\tau A) &= 0
\end{align}

where $\tau_A$ is her proper time. Diana’s departure point, trajectory, and proper acceleration remain unspecified for now.
\end{framed}

\begin{framed}
\textbf{(a)}
Before attempting any experiments, Artemis constructs a new coordinate system $(t, x, y, z)$ for making measurements in the neighbourhood of her rocket ship. The construction proceeds in three steps, which we copy here. 
\end{framed}

\begin{framed}
\textbf{i.} 
Fermi-Walker transport the Minkowski basis vectors $\vec{e}_0$, $\vec{e}_1$, $\vec{e}_2$, and $\vec{e}_3$ along Artemis’s world line. Show that this procedure yields a new basis
\begin{align}
\vec{e}_0(\tau_A) &= \vec{e}_0 \cosh g\tau_A +\vec{e}_1 \sinh g\tau_A,\label{ai. eq1}\\
\vec{e}_1(\tau_A) &= \vec{e}_0 \sinh g\tau_A +\vec{e}_1 \cosh g\tau_A,\\
\vec{e}_2(\tau_A) &= \vec{e}_2,\\
\vec{e}_3(\tau_A) &= \vec{e}_3\label{ai. eq4}
\end{align}

at the point along the world line labelled by $\tau_A$.
\end{framed}

\begin{framed}
\textbf{ii.} 
Verify that one can obtain (\ref{ai. eq1})-(\ref{ai. eq4}) by Lorentz boosting the Minkowski basis vectors into the rocket ship’s momentarily comoving reference frame.
\end{framed}

\begin{framed}
\textbf{iii.} 
Let the vector $x'\vec{e}_{1'}+y'\vec{e}_{2'}+z'\vec{e}_{3'}$ be the displacement of an event $P$ from the spacetime location of the centre of mass of Artemis’s rocket ship. Let $t'$ be the proper time measured by Artemis at the space point $(x,y,z)=(0,0,0)$. Show that these definitions lead to the transformation
\begin{align}
t&=(g^{-1}+x')\sinh gt',\\
x&=(g^{-1}+x') \cosh gt',\\
y&=y'\\
z&=z',
\end{align}
if $P$ occurs at $(t,x,y,z)$ in Minkowski coordinates.
\end{framed}




\end{document}







