\documentclass[a4paper]{article} % A4 paper and 11pt font size

\usepackage{braket}
\usepackage{amsmath}
\usepackage{amssymb}
\usepackage{bm}
\usepackage[utf8]{inputenc}
\usepackage{verbatim}
\usepackage{tikz}
%\usepackage{tikz-feynman}
\usepackage{pgfplots}
\usepackage{pgffor}
\usepackage[version-1-compatibility]{siunitx}
\usepackage{fancyhdr}

\usepackage{hyperref}
\usepackage{geometry}

 \geometry{
 a4paper,
 total={210mm,297mm},
 left=28mm,
 right=28mm,
 top=30mm,
 bottom=40mm,
 }


\usepackage{framed}
\usepackage{amssymb} %for Lagrangian L, order O
\usepackage{cancel} %for strikethroughs
\usepackage{slashed} %for Feynman slashes

\newcommand{\pmx}[1]{\begin{pmatrix}#1\end{pmatrix}}

\usepackage{gensymb}

\usepackage{fancyhdr}
\usepackage{pdflscape}
\usepackage{bm}

%for side-by-side figures
\usepackage{graphicx}
\usepackage{caption}
\usepackage{subcaption}

\setlength{\parindent}{2em}
\setlength{\parskip}{1em}
\renewcommand{\baselinestretch}{1.1}

%----------------------------------------------------------------------------------------
%	TITLE SECTION
%----------------------------------------------------------------------------------------
\setlength\parindent{0pt} % Removes all indentation from paragraphs - comment this line for an assignment with lots of text


\pagenumbering{arabic}
\begin{document}
\pagestyle{empty}

\newcommand{\HRule}{\rule{\linewidth}{0.5mm}}

\begin{titlepage}

    \begin{center}
        \textsc{\large SN: 587623}\\[6cm]

        \HRule \\[0.5cm]
		\Huge \textbf{PHYC90012 General Relativity}\\[0.5cm]
        \huge \textbf{Assignment 2}\\[0.5cm] 
        \HRule \\[1.5cm]
        \begin{minipage}{0.4\textwidth}
        \begin{center}

        \large By \\[0.75cm]
        \huge Braden \scshape Moore \\[0.5cm]
        \normalsize \normalfont Master of Science \\
        The University of Melbourne \\

        \end{center}
        \end{minipage}

        \vfill

        \large \today
    \end{center}

\newpage
\end{titlepage}
%----------------------------------------------------------------------------------------
\pagestyle{fancy}
\pagenumbering{arabic}
\rfoot{\textsc{Braden Moore, 587623}}
\lfoot{\textsc{\today}}
\rhead{\textsc{General Relativity: Assignment 2}}
\setcounter{page}{1}
\setcounter{section}{1}
\section{Klein's geometry}
%%%%%%%%%%%%%%%%%%%%QUESTION 1%%%%%%%%%%%%%%%%%%%%%
\begin{framed}
A two-dimensional surface is covered by coordinates $(u,v)$ in the domain $u^2+v^2=1$. The independent components of the metric are given by
\begin{align}
g_{11}&=\frac{a^2(1-v^2)}{(1-u^2-v^2)^2},\label{1eq1}\\
g_{12}&=\frac{a^2 uv}{(1-u^2-v^2)^2},\\
g_{22}&=\frac{a^2(1-u^2)}{(1-u^2-v^2)^2},\label{1eq3}
\end{align}
the independent components of the inverse metric are given by
\begin{align}
g^{11}&=a^{-2}(1-u^2)(1-u^2-v^2),\\
g^{12}&=-a^{-2} uv (1-u^2-v^2),\\
g^{22}&= a^{-2}(1-v^2)(1-u^2-v^2),\label{1eq6}
\end{align}
and the independent, non-zero Christoffel symbols are given by
\begin{align}
\Gamma^{1}_{11}&=\frac{2u}{1-u^2-v^2},\label{gamma1 11}\\
\Gamma^{1}_{12}&=\frac{v}{1-u^2-v^2},\\
\Gamma^{2}_{12}&=\frac{u}{1-u^2-v^2},\\
\Gamma^{2}_{22}&=\frac{2v}{1-u^2-v^2}.
\end{align}
Remember that $g_{\alpha\beta}$, $g^{\alpha\beta}$, and $\Gamma^{\lambda}_{\alpha\beta}$ are all symmetric in $\alpha$ and $\beta$.
\end{framed}

\begin{framed}
\textbf{(a)} Starting from~(\ref{1eq1})-(\ref{1eq6}), derive the expression~(\ref{gamma1 11}) for $\Gamma^{1}_{11}$.
\end{framed}

We begin with the expression for the Christoffel symbols in terms of the metric
\begin{equation}
\Gamma^{\lambda}_{\alpha \beta}=\frac{1}{2}g^{\lambda\mu}\left(g_{\mu\alpha,\beta}+g_{\mu\beta,\alpha}
-g_{\alpha\beta,\mu}\right)
\end{equation}

We now calculate the values of $g_{\alpha\beta,\mu}$ from~(\ref{1eq1})-(\ref{1eq3})
\begin{align}
g_{11,1}&=\frac{\partial g_{11}}{\partial x^1}=\frac{\partial\left(\frac{a^{2}(1-v^2)}{(1-u^2-v^2)^2}\right)}{\partial u}\\
&=\frac{4a^2u(1-v^2)}{(1-u^2-v^2)^3}\\
g_{12,1}&=\frac{\partial g_{12}}{\partial x^1}=\frac{\partial\left(\frac{a^{2}uv}{(1-u^2-v^2)^2}\right)}{\partial u}\\
&=\frac{a^2 v(3u^2-v^2+1)}{(1-u^2-v^2)^3}\\
&=g_{21,1}\quad\text{by symmetry}\\
g_{22,1}&=\frac{\partial g_{22}}{\partial x^1}=\frac{\partial\left(\frac{a^{2}(1-u^2)}{(1-u^2-v^2)^2}\right)}{\partial u}\\
&=\frac{2a^2u(1-u^2+v^2)}{(1-u^2-v^2)^3}
\intertext{By inspecting the components of the metric above, we see that $g_{11,1}\mapsto g_{22,2}$ with $u\leftrightarrow v$, similarly $g_{12,1}\mapsto g_{12,2}$ with $u\leftrightarrow v$, and $g_{11,2}\mapsto g_{22,1}$ with $u\leftrightarrow v$. Hence}
g_{11,2}&=\frac{2a^2v(1-v^2+u^2)}{(1-u^2-v^2)^3}\\
g_{12,2}&=\frac{a^2 u(3v^2-u^2+1)}{(1-u^2-v^2)^3}\\
&=g_{21,2}\quad\text{by symmetry}\\
g_{22,2}&=\frac{4a^2v(1-u^2)}{(1-u^2-v^2)^3}
\end{align}

So now we can evaluate the Christoffel symbols.
\begin{align}
\Gamma^{1}_{11}&=\frac{1}{2}g^{1\mu}\left(g_{\mu 1,1}+g_{\mu 1,1}-g_{11,\mu}\right)\\
&=\frac{1}{2}\left[g^{11}g_{11,1}+g^{12}\left(2g_{21,1}-g_{11,2}\right)\right]\\
&=\frac{1}{2}\left[a^{-2}(1-u^2)(1-u^2-v^2)\frac{4a^2u(1-v^2)}{(1-u^2-v^2)^3}\right.\nonumber\\
&~+\left. -a^{-2}uv(1-u^2-v^2)\left(\frac{2a^2v(3u^2-v^2+1)-2a^2v(1-v^2+u^2)}{(1-u^2-v^2)^3}\right)\right]\\
&=\frac{1}{2}\left[\frac{(1-u^2)\cdot 4u(1-v^2)-uv\cdot 2v(3u^2-u^2)}{(1-u^2-v^2)^2}\right]\\
&=\frac{1}{2}\left[\frac{4u(1-u^2)(1-v^2)-4u^2v^2}{(1-u^2-v^2)^2}\right]\\
&=2u\left[\frac{1-u^2-v^2+\cancel{u^2v^2-u^2v^2}}{(1-u^2-v^2)^2}\right]\\
&=\frac{2u}{1-u^2-v^2}
\intertext{as required.}
\Gamma^{1}_{12}&=\frac{1}{2}g^{1\mu}\left(g_{\mu 1,2}+g_{\mu 2,1}-g_{12,\mu}\right)\\
&=\frac{1}{2}\left[g^{11} g_{11,2}+g^{12}g_{22,1}\right]\\
&=\frac{1}{2}\left[a^{-2}(1-u^2)(1-u^2-v^2)\frac{2a^2v(1-v^2+u^2)}{(1-u^2-v^2)^3}\right.\nonumber\\
&~+\left. -a^{-2}uv(1-u^2-v^2)\frac{2a^2u (1-u^2+v^2)}{(1-u^2-v^2)^3}\right]\\
&=\frac{1}{2}\left[\frac{(1-u^2)\cdot 2v(1-v^2-u^2)-uv\cdot 2u(1-u^2+v^2)}{(1-u^2-v^2)^2}\right]\\
&=v\left[\frac{(1-u^2)(1-v^2+u^2)-u^2(1-u^2+v^2)}{(1-u^2-v^2)^2}\right]\\
&=v\left[\frac{\cancel{1-u^2-v^2}}{(1-u^2-v^2)^{\cancel{2}}}\right]\\
&=\frac{v}{1-u^2-v^2}
\intertext{as required.}
\Gamma^{2}_{12}&=\frac{1}{2}g^{2\mu}\left(g_{\mu 1,2}+g_{\mu 2,1}-g_{12,\mu}\right)\\
&=\frac{1}{2}\left[g^{21}g_{11,2}+g^{22}g_{22,1}\right]
\intertext{Now, we deduced earlier that $g_{11,2}=g_{22,1}\rvert_{u\leftrightarrow v}$, we see by inspection of~(\ref{1eq6}) that $g^{22}=g^{11}\rvert_{u\leftrightarrow v}$, and by symmetry of the metric we have $g^{21}=g^{12}$. We note also that $g^{12}$ is symmetric under interchange of $u$ and $v$. Combining these results we find}
\Gamma^{2}_{12}&=\frac{1}{2}\left[g^{12}g_{22,1}+g^{11}g_{11,2}\right]\rvert_{u\leftrightarrow v}\\
&=\Gamma^{1}_{12}\rvert_{u\leftrightarrow v}\\
&=\frac{u}{1-u^2-v^2}
\end{align}



\pagebreak %-----------------------------------------------------------

\begin{framed}
\textbf{(b)} Prove that the Riemann tensor with all indices lowered, $R_{\alpha\beta\gamma\delta}$, contains four nonzero elements, any three of which can be written in terms of the fourth.
\end{framed}


\pagebreak %-----------------------------------------------------------

\begin{framed}
\textbf{(c)} Prove that, in Klein’s geometry, the Ricci tensor satisfies
\begin{align}
R_{\alpha\beta}&=-\frac{g_{\alpha\beta}}{a^2},
\intertext{and the Ricci scalar satifies}
R&=-\frac{2}{a^2}.
\end{align}
\end{framed}

\pagebreak %-----------------------------------------------------------

\begin{framed}
\textbf{(d)} Answer each of the following questions in one or two sentences.
\end{framed}

\begin{framed}
\textbf{i.} In what fundamental way does Klein’s geometry differ from a two-sphere?
\end{framed}

\pagebreak %-----------------------------------------------------------

\begin{framed}
\textbf{ii.} The hyperbola $x^2-y^2=1$ is rotated around the $y$-axis to form a three-dimensional hyperboloid of revolution. Does it possess positive or negative curvature? Justify your answer physically with a diagram; do not attempt to calculate anything.
\end{framed}

\pagebreak %-----------------------------------------------------------

\begin{framed}
\textbf{iii.} The hyperbola $x^2-y^2=1$ is now rotated around the $x$-axis. What is the sign of the curvature this time? Why?
\end{framed}

\pagebreak %-----------------------------------------------------------

\begin{framed}
\textbf{iv.} Setting aside their dimensionality, in what fundamental way do the hyperboloids of revolution in parts (d)(ii) and (d)(iii) differ from Klein's geometry? Justify your answer in words; don’t try to calculate anything.
\end{framed}

\pagebreak %-----------------------------------------------------------

\begin{framed}
\textbf{v.} Identify a spacetime manifold, that resembles Klein's geometry. Don’t worry too much about the precise mathematical meaning of ``resembles'', a qualitative justification is fine.
\end{framed}

\pagebreak %-----------------------------------------------------------

\begin{framed}
\textbf{(e)} Consider the triangle $\Delta$ABC, whose sides are ``straight lines'' (geodesics) joining the points A$(0,0)$, B$(b,0)$, and C$(0,b)$, with $b<1$. It is easy to show (you don't need to!) that the sides AB and AC are just the curves $v=0$ and $u=0$ respectively.
\end{framed}

\begin{framed}
\textbf{i.} What is the equation of the geodesic joining B and C?
\end{framed}

\pagebreak %-----------------------------------------------------------

\begin{framed}
\textbf{ii.} Prove that the sum of the interior angles of $\Delta$ABC is
\begin{equation}
\Sigma = \angle\text{ABC}+\angle\text{BCA}+\angle\text{CAB}=\frac{\pi}{2}+2\cos^{-1}\left(\frac{1}{\sqrt{2-b^2}}\right).
\end{equation}
The sum of the angles is less than 180 degrees!
\end{framed}

\pagebreak %-----------------------------------------------------------

\begin{framed}
\textbf{iii.} Triangles in Klein’s geometry can have $\sum=0$! Without proof, sketch what such a triangle might look like. Your sketch by necessity will be an incomplete representation; there is no way to draw a Klein triangle faithfully on a flat page.
\end{framed}

\pagebreak %-----------------------------------------------------------

\begin{framed}
\textbf{(f) i.} Write down a closed form expression for the area A of $\Delta$ABC as an integral over a subset of the $(u,v)$ domain.
\end{framed}

\pagebreak %-----------------------------------------------------------

\begin{framed}
\textbf{ii.} By changing variables to $y=v+u$ and $z=v-u$, recast your integral in the form
\begin{equation}
A=2a^2\int^b_0 \frac{dy~y}{(2-y^2)\sqrt{1-y^2}}.
\end{equation}
Hence show that one has
\begin{equation}
A=a^2(\pi-\Sigma).\label{A(pi,Sigma)}
\end{equation}
\end{framed}

\pagebreak %-----------------------------------------------------------

\begin{framed}
\textbf{iii.} Explain briefly, in one or two sentences, why~(\ref{A(pi,Sigma)}) guarantees the \emph{nonexistence} of similar triangles in Klein’s geometry.
\end{framed}

\pagebreak %-----------------------------------------------------------

\begin{framed}
\textbf{(g)} A vector $\vec{W}$ with equal components $W^1$ and $W^2$ at the point $A(0,0)$ is parallel transported along the geodesic AB. Show that its components, when it reaches the point B$(b,0)$, are in the ratio
\begin{equation}
\frac{W^1}{W^2}=(1-b^2)^{1/2}
\end{equation}
\end{framed}



\end{document}







