\documentclass[a4paper]{article} % A4 paper and 11pt font size

\usepackage{braket}
\usepackage{amsmath}
\usepackage{amssymb}
\usepackage{bm}
\usepackage[utf8]{inputenc}
\usepackage{verbatim}
\usepackage{tikz}
%\usepackage{tikz-feynman}
\usepackage{pgfplots}
\usepackage{pgffor}
\usepackage[version-1-compatibility]{siunitx}
\usepackage{fancyhdr}
\usepackage{mathtools} %for underbraces without whitespace (\mathclap{})


\usepackage{hyperref}
\usepackage{geometry}

 \geometry{
 a4paper,
 total={210mm,297mm},
 left=28mm,
 right=28mm,
 top=30mm,
 bottom=40mm,
 }


\usepackage{framed}
\usepackage{amssymb} %for Lagrangian L, order O
\usepackage{cancel} %for strikethroughs
\usepackage{slashed} %for Feynman slashes

\newcommand{\pmx}[1]{\begin{pmatrix}#1\end{pmatrix}}
\newcommand{\ph}[1]{\phantom{#1}}
\newcommand{\diag}{\text{diag}}

\newcommand{\ao}{A_{\text{out}}}
\newcommand{\ai}{A_{\text{in}}}

\usepackage{gensymb}

\usepackage{fancyhdr}
\usepackage{pdflscape}
\usepackage{bm}

%for side-by-side figures
\usepackage{graphicx}
\usepackage{caption}
\usepackage{subcaption}

\setlength{\parindent}{2em}
\setlength{\parskip}{1em}
\renewcommand{\baselinestretch}{1.1}


%----------------------------------------------------------------------------------------
%	TITLE SECTION
%----------------------------------------------------------------------------------------
\setlength\parindent{0pt} % Removes all indentation from paragraphs - comment this line for an assignment with lots of text


\pagenumbering{arabic}
\begin{document}
\pagestyle{empty}

\newcommand{\HRule}{\rule{\linewidth}{0.5mm}}

\begin{titlepage}

    \begin{center}
        \textsc{\large SN: 587623}\\[6cm]

        \HRule \\[0.5cm]
		\Huge \textbf{PHYC90012 General Relativity}\\[0.5cm]
        \huge \textbf{Assignment 4}\\[0.5cm] 
        \HRule \\[1.5cm]
        \begin{minipage}{0.4\textwidth}
        \begin{center}

        \large By \\[0.75cm]
        \huge Braden \scshape Moore \\[0.5cm]
        \normalsize \normalfont Master of Science \\
        The University of Melbourne \\

        \end{center}
        \end{minipage}

        \vfill

        \large \today
    \end{center}

\newpage
\end{titlepage}
%----------------------------------------------------------------------------------------
\pagestyle{fancy}
\pagenumbering{arabic}
\rfoot{\textsc{Braden Moore, 587623}}
\lfoot{\textsc{\today}}
\rhead{\textsc{General Relativity: Assignment 4}}
\setcounter{page}{1}
\setcounter{section}{3}
\section{Detecting gravitational waves with lasers}

\begin{framed}
A Michelson laser interferometer lies in the plane $z = 0$. Its arms are aligned with the $x$- and $y$-axes and have the same
length, $L$, in the absence of a gravitational wave. A weak gravitational wave is incident normally on the interferometer. In the transverse, traceless gauge, the perturbation to the background Minkowski metric $\eta_{\mu\nu}$ is given by the plane-wave form
\begin{equation}
h_{\mu\nu}=H_{\mu\nu}\exp(iKz-i \Omega t),
\end{equation}
with
\begin{equation}
H_{\mu\nu}=\pmx{0&0&0&0\\0&h_+&h_{\times}&0\\0&h_{\times}&-h_{+}&0\\0&0&0&0}.
\end{equation}
In this question, we assume that the gravitational wave consists purely of the plus polarisation, i.e. $h_{\times}=0$
\end{framed}

\begin{framed}
\textbf{(a)} Traditionally we analyse the phase shift in a gravitational wave interferometer using ray optics, i.e. by considering photon geodesics.
\end{framed}

\begin{framed}
\textbf{i.} In lectures we showed that the spatial coordinates of the interferometer's mirrors remain constant in the transverse, traceless gauge. Nevertheless the interferometer measures a phase shift between its two arms, when a gravitational wave interacts with the system. Explain in words why there is no contradiction.
\end{framed}

While the spatial separation between the arms are indeed constant in a local Minkowski space and in our transverse traceless gauge, this does not necessarily constrain the curvature of the space.

It is this curvature in the global frame, due to the presence of a gravitational wave, that causes a measurable phase shift between photons travelling in the arms; hence there is no contradiction.

\begin{framed}
\textbf{ii.} Let $[t(\lambda),x(\lambda),0,0]$ be the trajectory of a photon travelling along the x-axis, parametrised by an affine parameter $\lambda$. Show that, along the trajectory, one has
\begin{equation}
\frac{dt}{dx}=\pm\left[1+\frac{h_{+}}{2}\exp(-i\Omega t)\right]
\end{equation}
to first order in the small dimensionless quantity $h_+$.
\end{framed}

For a null ray (photon), $ds^2=0$. Also, in general $ds^2=g(d\vec{x},d\vec{x})=g_{\alpha\beta}dx^{\alpha}dx^{\beta}$. Noting that $g_{\alpha \beta}=\eta_{\alpha \beta}+h_{\alpha \beta}$, we find

\begin{align}
g_{\alpha\beta}dx^{\alpha}dx^{\beta}&=0\\
g_{00}(dx)^2 + g_{11}(dt)^2&=0\\
\Rightarrow \left(\frac{dt}{dx}\right)^2 &=-\frac{g_{00}}{g_{11}}
\intertext{We note the relevant metric components as}
g_{00} &= -1\\
g_{11}&= 1+ h_{+}\exp(-i\Omega t)
\intertext{hence we find}
\Rightarrow \left(\frac{dt}{dx}\right)^2 &= \left[1+h_+ \exp(-i\Omega t)\right]^{-1}\\
\Rightarrow \frac{dt}{dx} &= \pm \left[ 1 + h_+ \exp(-i\Omega t)\right]^{-1/2}\\
&= \pm \left[1+ \frac{h_+}{2}\exp(-i\Omega t)\right]
\end{align}
to first order in $h_+$ (we can Taylor expand since $h_+ \ll 1$).


\begin{framed}
\textbf{iii.} Show that the round-trip travel time along the x-axis is
\begin{equation}
\Delta t=2L + \frac{h_+}{\Omega}\sin(\Omega L) \exp(-i\Omega L) \label{eq4}
\end{equation}
to first order in $h_+$. Identify and justify any approximations you make.
\end{framed}

Starting from~(\ref{eq4}),
\begin{equation}
\frac{dt}{dx}=\left[1+\frac{h_+}{2}\exp(-i\Omega t)\right]
\end{equation}
\begin{align}
\Rightarrow dt~\left[1+\frac{h_+}{2}\exp(-i\Omega t)\right]^{-1}&=dx
\intertext{Since $h_+ \ll 1$, to first-order in $h_+$ we can expand the brackets of the left-hand side (LHS) so that}
\int dt~\left[1-\frac{h_+}{2}\exp(-i\Omega t)\right]&=\int dx\\
\Rightarrow t+\frac{h_+}{2}\frac{1}{i\Omega}\exp(-i\Omega t)&=x+c
\end{align}
where $c$ is some constant.
\begin{align}
\Rightarrow t&=x+c-\frac{h_+}{2i\Omega}\exp(-i\Omega t)
\intertext{Using initial conditions $t=0, x=0$ we find}
c&=\frac{h_+}{2i\Omega}\exp(0)=\frac{h_+}{2i\Omega}\\
\Rightarrow t&=x+\frac{h_+}{2i\Omega}\left[1-\exp(-i\Omega t)\right]\\
&=x+\frac{h_+}{\Omega}\frac{1}{2i}\left[\exp\left(\frac{i\Omega t}{2}\right)-\exp\left(\frac{-i\Omega t}{2}\right)
\right]\exp\left(\frac{-i\Omega t}{2}\right)\\
&=x+\frac{h_+}{\Omega}\sin\left(\frac{\Omega t}{2}\right)\exp\left(\frac{-i\Omega t}{2}\right)
\end{align}
by using the identity $\sin(x)=\frac{1}{2i}\left[e^{x}-e^{-x}\right]$. To zeroth order, the round-trip distance is $x=2L$ and $t=x$; hence $t=2L$. Using these zeroth order approximations, we find
\begin{equation}
\Delta t = 2L + \frac{h_+}{\Omega}\sin(\Omega L) \exp(-i\Omega L)
\end{equation}



\begin{framed}
\textbf{iv.} Without repeating the calculation, explain why the wave-induced perturbation to the round-trip travel time along the other arm (i.e. for photons travelling along the y-axis) is equal but opposite to the result in~(\ref{eq4}). Hence the total phase shift observed is given by
\begin{equation}
\Delta \phi = 2\omega (\Delta t - 2L) = \frac{2h_+ \omega}{\Omega}\sin(\Omega L)\exp(-i\Omega L),\label{eq5}
\end{equation}
where $\omega$ is the photon frequency.
\end{framed}

The $yy$ component of the perturbation $h_{\mu\nu}$ is proportional to $-h_+$, whereas the $xx$ component is proportional to $h_+$. From this, we can expect
\begin{equation}
\frac{dt}{dx}=\pm\left[1-\frac{h_+}{2}\exp(-i\Omega t)\right]
\end{equation}
by inspecting the expression for $\frac{dt}{dx}$. Following this, we would expect the round-trip travel time to include $-h_+$ where we have $+h_+$ in the $x$ case;
\begin{equation}
\Delta t_{y} = 2L  - \frac{h_+}{\Omega}\sin(\Omega L)\exp(-i\Omega L)
\end{equation}


\begin{framed}
\textbf{(b)} We now analyse the phase shift using physical optics, i.e. by solving for the electromagnetic fields of the laser in the curved spacetime of the gravitational wave. If we are ``lucky'',  we should get the same answer for $\Delta \phi$!
\end{framed}

\begin{framed}
\textbf{i.} The inhomogeneous Maxwell equations (Gauss, Ampere) in vacuum take the exact form
\begin{equation}
F^{\mu\nu}_{\ph{\mu\nu},\nu}=0 \label{eq6}
\end{equation}
in flat spacetime, where $F_{\mu\nu}=\partial_{\mu} A_{\nu} - \partial_{\nu} A_{\mu}$ is the Faraday tensor, and $A$ is the electromagnetic potential. Argue carefully that, in the curved spacetime of the gravitational wave, equation~(\ref{eq6}) takes the exact form
\begin{equation}
F^{\mu\nu}_{\ph{\mu\nu};\nu}=0
\end{equation}
and hence that we have
\begin{equation}
F^{\mu\nu}_{\ph{\mu\nu},\nu}=-\Gamma^{\mu}_{\ph{\mu}\lambda\nu} F^{\lambda\nu} - \Gamma^{\nu}_{\ph{\nu}\lambda\nu}
F^{\mu\lambda},\label{eq8}
\end{equation}
where $\Gamma^{\mu}_{\lambda\nu}$ is a Christoffel symbol.
\end{framed}

In flat spacetime, we have $F^{\mu\nu}_{\ph{\mu\nu},\nu}=0$. The LHS of this equation comprises components of the tensor $\nabla F$. However, we know that $F^{\mu\nu}_{\ph{\mu\nu};\nu}$ are also components of the same tensor in arbitrary coordinates! So in these arbitrary coordinates, we know $F^{\mu\nu}_{\ph{\mu\nu};\nu}=0$. 

This is a ``good'' tensor equation; it is true in \underline{all} coordinate systems, so in the curved spacetime of the gravitational wave we have

\begin{equation}
F^{\mu\nu}_{\ph{\mu\nu};\nu}=0
\end{equation}


\begin{framed}
\textbf{ii.} Show that~(\ref{eq8}) reduces to
\begin{equation}
F^{\mu\nu}_{\ph{\mu\nu},\nu}=0 \label{eq9}
\end{equation}
to first order in $h_+$. You may find it useful to apply the identity
\begin{equation}
\Gamma^{\nu}_{\ph{\nu}\lambda\nu}=\frac{(\sqrt{-g})_{,\lambda}}{\sqrt{-g}},
\end{equation}
where $g$ is the metric determinant, but there are many other valid approaches as well.
\end{framed}

We shall first set about calculating the second Christoffel term, using the relationship
\begin{equation}
\Gamma^{\nu}_{\ph{\nu}\lambda\nu}=\frac{(\sqrt{-g})_{,\lambda}}{\sqrt{-g}}
\end{equation}

We note that the determinant of a diagonal matrix is simply the product of its diagonal components, so for the matrix
\begin{equation}
g_{\alpha\beta}=\diag(-1,1+h_+\exp(iKz-i\Omega t),1-h_+\exp(iKz-i\Omega t),1)
\end{equation}
we calculate the metric determinant $g$ as
\begin{align}
g&=-\big(1-\left[h_+\exp(iKz-i\Omega t)\right]^2\big)\\
\Rightarrow \sqrt{-g}&=\sqrt{1-h_+^2\exp(2iKz-2i\Omega t)}\\
&= 1-\frac{1}{2}h_+^2 \exp(2iKz - 2i\Omega t)\\
&=1
\end{align}
to first-order in $h_+$

So now we can calculate $\Gamma^{\nu}_{\ph{\nu}\lambda\nu}$. Observing that $\sqrt{-g}$ is a constant, we immediately find
\begin{equation}
\Gamma^{\nu}_{\ph{\nu}\lambda\nu}=0\label{sqrt g}
\end{equation}
for all $\lambda$.

Now we consider the first term of~(\ref{eq8}), given as $\Gamma^{\mu}_{\ph{\mu}\lambda\nu} F^{\lambda\nu}$. From the definition of $F^{\mu\nu}$, we can see that it is antisymmetric.
\begin{align}
F^{\mu\nu}&=\partial^{\mu}A^{\nu}-\partial^{\nu}A^{\mu}\\
F^{\nu\mu}&=\partial^{\nu}A^{\mu}-\partial^{\mu}A^{\nu}\\
&=-\left(\partial^{\mu}A^{\nu}-\partial^{\nu}A^{\mu}\right)\\
\Rightarrow F^{\nu\mu}&=-F^{\mu\nu}
\end{align}

If $\Gamma^{\mu}_{\ph{\mu}\lambda\nu}$ is symmetric in  $\lambda,\nu$ (that is, $\Gamma^{\mu}_{\ph{\mu}\lambda\nu}
=\Gamma^{\mu}_{\ph{\mu}\nu\lambda}$), then $\Gamma^{\mu}_{\ph{\mu}\lambda\nu}F^{\lambda\nu}$ will be zero, since
\begin{align}
\Gamma^{\mu}_{\ph{\mu}\lambda\nu}F^{\lambda\nu}&=-\Gamma^{\mu}_{\ph{\mu}\nu\lambda}F^{\nu\lambda}\\
&=-\Gamma^{\mu}_{\ph{\mu}\lambda\nu}F^{\lambda\nu}
\intertext{under relabelling,}
\Rightarrow \Gamma^{\mu}_{\ph{\mu}\lambda\nu}F^{\lambda\nu}&=0\label{symm}
\end{align}

We will show that $\Gamma^{\mu}_{\ph{\mu}\lambda\nu}$ is indeed symmetric in $\lambda,\nu$.

The non-zero components of the metric $g^{\alpha\beta}$ are
\begin{align}
g^{00}&=-1\\
g^{11}&=1+h_+\exp(iKz-i\Omega t)\\
g^{22}&=1-h_+\exp(iKz-i\Omega t)\\
g^{33}&=1
\end{align}

From above, we see that the only non-zero terms $g_{\alpha\beta,\mu}$ are
\begin{align}
g_{11,0}&=-i\Omega h_+ \exp(iKz-i\Omega t)\label{g1}\\
g_{11,3}&=i K h_+ \exp(iKz-i\Omega t)\\
g_{22,0}&=i \Omega h_+ \exp(iKz-i\Omega t)\\
g_{22,3}&=-i K h_+ \exp(iKz-i\Omega t)\label{g4}
\end{align}

The components of $\Gamma^{\mu}_{\ph{\mu}\lambda\nu}$ can be calculated as
\begin{align}
\Gamma^{\mu}_{\ph{\mu}\lambda\nu}&=\frac{1}{2}g^{\mu\alpha}\left(g_{\alpha\lambda,\nu}
+g_{\alpha\nu,\lambda}-g_{\lambda\nu,\alpha}\right)\\
&=\frac{1}{2}\left[g^{\mu 0}\left(\cancel{g_{0\lambda,\nu}}+\cancel{g_{0\nu,\lambda}}-g_{\lambda\nu,0}\right)
+g^{\mu 1}\left(g_{1\lambda,\nu}+g_{1\nu,\lambda}-\cancel{g_{\lambda\nu,1}}\right)
+g^{\mu 2}\left(g_{2\lambda,\nu}+g_{2\nu,\lambda}-\cancel{g_{\lambda\nu,2}}\right)\right.\nonumber\\
&\qquad\left. +g^{\mu 3}\left(\cancel{g_{3\lambda,\nu}}+\cancel{g_{3\nu,\lambda}}-g_{\lambda\nu,3}\right)\right]\\
&=\frac{1}{2}\left[-g^{\mu 0}g_{\lambda\nu,0}+g^{\mu 1}\left(g_{1\lambda,\nu}+g_{1\nu,\lambda}\right)
+g^{\mu 2}\left(g_{2\lambda,\nu}+g_{2\nu,\lambda}\right)-g^{\mu 3}g_{\lambda\nu,3}\right]\label{chris}\\
\Gamma^{0}_{\ph{0}\lambda\nu}&=-\frac{1}{2}g^{00}g_{\lambda\nu,0}\\
&=\frac{1}{2}g_{\lambda\nu,0}
\intertext{From~(\ref{g1})-(\ref{g4}) we see that $g_{\lambda\nu,0}$ is symmetric in $\lambda,\nu$. Hence $\Gamma^{0}_{\ph{0}\lambda\nu}$ is symmetric in $\lambda,\nu$. Similarly for $\Gamma^{3}_{\ph{3}\lambda\nu}$,}
\Gamma^{3}_{\ph{3}\lambda\nu}&=-\frac{1}{2}g^{33}g_{\lambda\nu,3}\\
&=\frac{1}{2}g_{\lambda\nu,3}
\intertext{From~(\ref{g1})-(\ref{g4}) we see that $g_{\lambda\nu,3}$ is symmetric in $\lambda,\nu$, so that $\Gamma^{3}_{\ph{3}\lambda\nu}$ is symmetric in $\lambda,\nu$.}
\Gamma^{1}_{\ph{1}\lambda\nu}&=\frac{1}{2}\g^{11}\left(g_{1\lambda,\nu}+g_{1\nu,\lambda}\right)
\intertext{We see the only non-zero $\Gamma^{1}_{\ph{1}\lambda\nu}$ terms are $\Gamma^{1}_{\ph{1}10},\Gamma^{1}_{\ph{1}01},\Gamma^{1}_{\ph{1}31},\Gamma^{1}_{\ph{1}13}$.}
\Gamma^{1}_{\ph{1}10}&=\frac{1}{2}g^{11}g_{11,0}\\
\Gamma^{1}_{\ph{1}01}&=\frac{1}{2}g^{11}g_{11,0}=\Gamma^{1}_{\ph{1}10}\\
\Gamma^{1}_{\ph{1}13}&=\frac{1}{2}g^{11}g_{11,3}\\
\Gamma^{1}_{\ph{1}31}&=\frac{1}{2}g^{11}g_{11,3}=\Gamma^{1}_{\ph{1}13}
\intertext{Hence  $\Gamma^{1}_{\ph{1}\lambda\nu}$ is symmetric in $\lambda,\nu$. Similarly for $\Gamma^{2}_{\ph{2}\lambda\nu}$,}
\Gamma^{2}_{\ph{2}\lambda\nu}&=\frac{2}{2}\g^{22}\left(g_{2\lambda,\nu}+g_{2\nu,\lambda}\right)
\intertext{We see the only non-zero $\Gamma^{2}_{\ph{2}\lambda\nu}$ terms are $\Gamma^{2}_{\ph{2}20},\Gamma^{2}_{\ph{2}02},\Gamma^{2}_{\ph{2}32},\Gamma^{2}_{\ph{2}23}$.}
\Gamma^{2}_{\ph{2}20}&=\frac{2}{2}g^{22}g_{22,0}\\
\Gamma^{2}_{\ph{2}02}&=\frac{2}{2}g^{22}g_{22,0}=\Gamma^{2}_{\ph{2}20}\\
\Gamma^{2}_{\ph{2}23}&=\frac{2}{2}g^{22}g_{22,3}\\
\Gamma^{2}_{\ph{2}32}&=\frac{2}{2}g^{22}g_{22,3}=\Gamma^{2}_{\ph{2}23}
\end{align}
Hence  $\Gamma^{2}_{\ph{2}\lambda\nu}$ is symmetric in $\lambda,\nu$. Thus we have shown that $\Gamma^{\mu}_{\ph{\mu}\lambda\nu}$ is symmetric in $\lambda,\nu$. Combining this result with~(\ref{symm}) and~(\ref{sqrt g}), we conclude that
\begin{equation}
F^{\mu\nu}_{\ph{\mu\nu},\nu}=0 \label{eq9}
\end{equation}
to first order in $h_+$, as required!


\begin{framed}
\textbf{iii.} In flat space, equation~(\ref{eq9}) reduces to $\hat{D}A^\mu$ in the electromagnetic Lorenz gauge, where $\hat{D}=\eta^{\alpha\beta}\partial_{\alpha}\partial_{\beta}$ symbolises the d'Alembertian (wave) operator. In the curved space of the gravitational wave, show that~(\ref{eq9}) reduces to
\begin{equation}
\hat{D}A^{\mu}=-h^{\mu\alpha}_{\ph{\mu\alpha},\nu}F_{\alpha}^{\ph{\alpha}\nu}-h^{\nu\alpha}_{\ph{\nu\alpha},\nu}
F^{\mu}_{\ph{\mu}\alpha}+h^{\mu\alpha} F_{\alpha\nu}^{\ph{\alpha\nu},\nu}+h^{\nu\alpha} F^{\mu}_{\ph{\mu}\alpha,\nu}\label{eq11}
\end{equation}
up to first order in $h_+$, where all indices are raised and lowered with the Minkowski metric in~(\ref{eq11}) and henceforth.
\end{framed}

Beginning with~(\ref{eq9}), we have
\begin{align}
0&=\left(F^{\mu\nu}_{\ph{\mu\nu},\nu}\right)_{,\nu}\\
&=\left(g^{\alpha\mu}g^{\beta\nu}F_{\alpha\beta}\right)_{,\nu}\\
&=\left[\left(\eta^{\alpha\mu}\eta^{\beta\nu}+\eta^{\alpha\mu}h^{\beta\nu}+\eta^{\beta\nu}h^{\alpha\mu}
+\cancel{h^{\alpha\mu}h^{\beta\nu}}\right)F_{\alpha\beta}\right]_{,\nu}
\intertext{where we can cancel the $h^2$ term as we are only considering first-order in $h_+$.}
\Rightarrow 0&=\left[\partial^{\mu}A^{\nu}-\partial^{\nu}A^{\mu}+h^{\beta\nu}\left(\partial^{\mu}A_{\beta}
-\partial_{\beta}A^\mu\right) +h^{\alpha\mu}\left(\partial_\alpha A^\nu - \partial^\nu A_\alpha\right)\right]_{,\nu}\\
&=\cancel{\partial_\nu \partial^\mu A^\nu} - 
\underbrace{\partial_\nu \partial^\nu A^\mu}_{\mathclap{=\hat{D}A^\mu}}
+h^{\beta\nu}_{\ph{\beta\nu},\nu}\left(\partial^\mu A_{\beta} - \partial_\beta A^\mu\right)
+h^{\beta\nu}\partial_\nu \left(\partial^\nu A_\beta - \partial_\beta A^\mu \right)\nonumber\\
&\qquad + h^{\alpha\mu}_{\ph{\alpha\mu},\nu}\left(\partial_{\alpha} A^\nu - \partial^\nu A_\alpha\right)
+h^{\alpha\mu}\partial_{\nu}\left(\partial_\alpha A^\nu - \partial^\nu A_\alpha\right)\\
\Rightarrow \hat{D}A^\mu&=
-h^{\beta\nu}_{\ph{\beta\nu},\nu}\left(\partial_\beta A^\mu - \partial^\mu A_{\beta}\right)
-h^{\alpha\mu}_{\ph{\alpha\mu},\nu}\left(\partial^\nu A_\alpha - \partial_{\alpha} A^\nu \right)
+h^{\beta\nu}\partial_\nu \left(\partial^\mu A_\beta - \partial_\beta A^\mu \right)
+h^{\alpha\mu}\partial_{\nu}\left(\partial_\alpha A^\nu - \partial^\nu A_\alpha\right)\\
&=-h^{\beta\nu}_{\ph{\beta\nu},\nu}F^{\mu}_{\ph{\mu}\beta}-h^{\alpha\mu}_{\ph{\alpha\mu},\nu}
F_{\alpha}^{\ph{\alpha}\nu}+h^{\beta\nu}F^{\mu}_{\ph{\mu}\beta,\nu}
+h^{\alpha\mu}F_{\alpha,\nu}^{\ph{\alpha,\nu}\nu}
\intertext{Following some rearrangement and relabelling, this becomes}
\hat{D}A^{\mu}&=-h^{\mu\alpha}_{\ph{\mu\alpha},\nu}F_{\alpha}^{\ph{\alpha}\nu}-h^{\nu\alpha}_{\ph{\nu\alpha},\nu}
F^{\mu}_{\ph{\mu}\alpha}+h^{\mu\alpha} F_{\alpha,\nu}^{\ph{\alpha\nu}\nu}+h^{\nu\alpha} F^{\mu}_{\ph{\mu}\alpha,\nu}
\end{align}
which is (more-or-less) as required!

\begin{framed}
\textbf{(c)} Consider a classical laser field propagating along the $x$-axis and polarised linearly along the $z$-axis, i.e. $\vec{A}=A^3 (t,x) \vec{e}_3$.
\end{framed}

\begin{framed}
\textbf{i.} Show that equation~(\ref{eq11}) reduces to
\begin{equation}
\left(-\frac{\partial^2}{\partial t^2}+\frac{\partial^2}{\partial x^2}\right)A^3 = -h_+ \exp(-i\Omega t) \frac{\partial^2 A^3}{\partial x^2}.\label{eq12}
\end{equation}
\end{framed}

Since we have
\begin{equation}
\vec{A}=A^3(t,x)\vec{e}_3 = (0,0,0,A)
\end{equation}
we see immediately that
\begin{equation}
\hat{D}A^\mu = 0\quad\text{for all }\mu\neq 3
\end{equation}
Combining this result with~(\ref{eq11}) we have
\begin{align}
\hat{D}A^3&=-\underbrace{h^{3\alpha}_{\ph{3\alpha},\nu}}_{=0}F_{\alpha}^{\ph{\alpha}\nu}-h^{\nu\alpha}_{\ph{\nu\alpha},\nu}
F^{3}_{\ph{3}\alpha}+\underbrace{h^{3\alpha}}_{=0}F_{\alpha\nu}^{\ph{\alpha\nu},\nu}
+h^{\nu\alpha}F^{3}_{\ph{3}\alpha,\nu}
\intertext{recalling that $h^{3\alpha}=h^{\alpha 3}=0$.}
\Rightarrow \hat{D}A^3&= -h^{\nu\alpha}_{\ph{\nu\alpha},\nu}F^{3}_{\ph{3}\alpha}
+h^{\nu\alpha}F^{3}_{\ph{3}\alpha,\nu}
\end{align}

We will now consider the LHS.
\begin{equation}
\hat{D}A^3(t,x)=\eta^{\alpha\beta}\partial_{\alpha}\partial_{\beta}A^3
\end{equation}
Since $A^3$ is a function of only $t$ and $x$, terms including $\partial_2$ and $\partial_3$ will be zero.
\begin{equation}
\hat{D}A^3=\eta^{00}\partial_{00}A^3 + \eta^{01}\partial_{01}A^3 + \eta^{10}\partial_{10}A^3
+\eta^{11}\partial_{11}A^3
\end{equation}
However we recall the components of the Minkowski metric,
\begin{align}
\eta^{00}&=-1\\
\eta^{11}&=1\\
\eta^{01}&=0\\
\eta^{10}&=0
\end{align}
so that the LHS becomes
\begin{equation}
\text{LHS}=-\partial_{00}A^3 + \partial_{11}A^3 = -\frac{\partial^2}{\partial t^2}A^3 + \frac{\partial^2}{\partial x^2}A^3
=\left(-\frac{\partial^2}{\partial t^2}+\frac{\partial^2}{\partial x^2}\right)A^3
\end{equation}

We consider now the right-hand side (RHS),
\begin{equation}
\text{RHS}=-h^{\nu\alpha}_{\ph{\nu\alpha},\nu}F^{3}_{\ph{3}\alpha}
+h^{\nu\alpha}F^{3}_{\ph{3}\alpha,\nu}
\end{equation}
Noting that the only non-zero components of $h^{\alpha\beta}$ are $h^{11}$ and $h^{22}$, we have
\begin{equation}
\text{RHS}=-h^{11}_{\ph{11},1}F^{3}_{\ph{3}1}-h^{22}_{2}F^3_{\ph{3}2}+h^{11}F^{3}_{\ph{3}1,1}
+h^{22}F^{3}_{\ph{3}2,2}
\end{equation}

Now calculating each of the terms,
\begin{align}
h^{11}_{\ph{11},1}&=\frac{\partial}{\partial x}\left[h_+\exp(iKz-i\Omega t)\right]=0\\
h^{22}_{\ph{22},2}&=\frac{\partial}{\partial y}\left[-h_+\exp(iKz - i\Omega t)\right]=0\\
F^{3}_{\ph{3}1,1}&=\partial_{1}\left(\cancel{\partial^3 A_1} - \partial_1 A^3\right)
=-\frac{\partial}{\partial x}\left(\frac{\partial}{\partial x}A^3\right)=-\frac{\partial^2 A^3}{\partial x^2}\\
F^{3}_{\ph{3}2,2}&=\partial_{2}\left(\cancel{\partial^3 A_2} - \cancel{\partial_2 A^3}\right)=0
\end{align}
Hence we conclude for the RHS,
\begin{equation}
\text{RHS}=-h^{11}\frac{\partial^2}{\partial x^2}A^3 = -h_+\exp(-i\Omega t) \frac{\partial^2 A^3}{\partial x^2}
\end{equation}

So we have proven that~(\ref{eq11}) reduces to
\begin{equation}
\left(-\frac{\partial^2}{\partial t^2}+\frac{\partial^2}{\partial x^2}\right)A^3 = -h_+ \exp(-i\Omega t) \frac{\partial^2 A^3}{\partial x^2}\label{a3 de}
\end{equation}
as required!

\begin{framed}
\textbf{ii.} By inspecting the mathematical form of~(\ref{eq12}) and noting that $h_+$ is small, explain why the laser field can be written approximately as a zeroth-order carrier wave emitted by the laser (constant amplitude $A_{\text{in}}$, frequency $\omega$, wavenumber $k$) plus a first-order correction generated by the interaction with the gravitational wave (slowly varying amplitude $A_{\text{out}}$, frequency $\omega+\Omega$, wavenumber $k$), viz.
\begin{equation}
A^3 = A_{\text{in}}\exp(-i\omega t + ikx) + A_{\text{out}}(t)\exp[-i(\omega + \Omega)t + ikx]. \label{eq13}
\end{equation}
\end{framed}

Taking~(\ref{a3 de}) to zeroth-order in $h_+$, we find the differential equation for $A^3$,
\begin{align}
\left(-\frac{\partial^2}{\partial t^2}+\frac{\partial^2}{\partial x^2}\right)A^3&=0\\
\Rightarrow \frac{\partial ^2 A^3}{\partial t^2}&=\frac{\partial^2 A^3}{\partial x^2}
\end{align}

The solution gives us $A^3$ in the form
\begin{equation}
A^3 = c_1 \exp(c_2 t + c_3 x)
\end{equation}
for constants $c_1, c_2, c_3$, where we necessarily require
\begin{equation}
c_2^2 = c_3^2.
\end{equation}
Since we have $\omega = k$, we take
\begin{align}
c_2^2&= -\omega ^2\\
\Rightarrow c_2 & = -i\omega
\intertext{where the negative sign is by convention.}
c_3^2&= -k^2\\
\Rightarrow c_3&=ik
\end{align}
where again, we choose the sign by convention. We also redefine the constant $c_1$ as
\begin{equation}
c_1 \equiv A_{\text{in}}
\end{equation}
where $A_{\text{in}}$ is the constant amplitude of the laser.

For the first-order correction, we begin by positing that $A^3$ has the form
\begin{equation}
A^3 = A_{\text{in}}\exp(-i\omega t + ikx)+f(t,x)\exp(-i\omega t + ikx)\label{a3 form}
\end{equation}
Substituting this into~(\ref{eq12}), and recalling $\left(-\frac{\partial^2}{\partial t^2}+\frac{\partial^2}{\partial x^2}\right)A_{\text{in}}\exp(-i\omega t + ikx)=0$, we have
\begin{equation}
\left(-\frac{\partial^2}{\partial t^2}+\frac{\partial^2}{\partial x^2}\right)f(t,x)\exp(-i\omega t + ikx)
=-h_+\exp(-i\Omega t)\frac{\partial^2}{\partial x^2}\left[A_{\text{in}}\exp(-i\omega t + ikx) + f(t,x)\exp(-i\omega t + ikx)\right]
\end{equation}
We assume that $f(t,x)$ has some dependence on $h_+$, so that to first order in $h_+$ we have
\begin{equation}
\left(-\frac{\partial^2}{\partial t^2}+\frac{\partial^2}{\partial x^2}\right)f(t,x)\exp(-i\omega t + ikx)
=-h_+\exp(-i\Omega t)\frac{\partial^2}{\partial x^2}\left[A_{\text{in}}\exp(-i\omega t + ikx)\right]
\end{equation}
\begin{equation}
\Rightarrow \left[-\frac{\partial^2 f}{\partial t^2}-\cancel{f(t,x)\times-\omega^2}+\frac{\partial^2 f}{\partial x^2}
+\cancel{f(t,x)\times -k^2}\right]\cancel{\exp(-i\omega t + ikx)}=h_+ k^2 \exp(-i\Omega t)A_{\text{in}}
\cancel{\exp(-i\omega t + ikx)}
\end{equation}
where we recall $\omega = k$.
\begin{equation}
\Rightarrow \left(-\frac{\partial^2}{\partial t^2}+\frac{\partial^2}{\partial x^2}\right)f(t,x)=h_+ k^2
\exp(-i\Omega t) A_{\text{in}}
\end{equation}
Since there is no $x$ dependence in the RHS, we will say that $f(t,x)=f(t)$, i.e. there is no $x$-dependence in $f$.
\begin{align}
\Rightarrow -\frac{\partial^2 f}{\partial t^2}&=-h_+ k^2 \exp(-i\Omega t)A_{\text{in}}\\
\frac{\partial f}{\partial t}&=\frac{1}{i\Omega}h_+ k^2 \exp(-i\Omega t)A_{\text{in}}+d_1\\
f(t)&=\frac{1}{\Omega^2}h_+k^2 \exp(-i\Omega t)A_{\text{in}}+d_1 t\\
\Rightarrow f(t)&=\exp(-i\Omega t)\left[\frac{1}{\Omega ^2}h_+ k^2 A_{\text{in}}+d_1 t \exp(i\Omega t)\right]
\end{align}
where $d_1$ is some constant. Hence we conclude,
\begin{equation}
f(t)=A_{\text{out}}(t)\exp(-i\Omega t)
\end{equation}
where $A_{\text{out}}$ is some time-varying amplitude. Finally, combining this with~(\ref{a3 form}) we can write $A^3$ in the form of a zeroth-order carrier wave plus a first-order correction as
\begin{equation}
A^3 = A_{\text{in}}\exp(-i\omega t + ikx) + A_{\text{out}}(t)\exp[-i(\omega + \Omega)t + ikx]
\end{equation}
as required!


\begin{framed}
\textbf{iii.} Use~(\ref{eq12}),~(\ref{eq13}), and the fact that $A_{\text{out}}(t)$ varies slowly ($\Omega \ll \omega$) to obtain
\begin{equation}
\frac{dA_{\text{out}}}{dt} - i\Omega A_{\text{out}} = -\frac{i h_+ k A_{\text{in}}}{2}.\label{eq14}
\end{equation}
\end{framed}

We begin by calculating $\dfrac{\partial^3 A^3}{\partial^2 t}$ and $\dfrac{\partial^3 A^3}{\partial^2 x}$ (note we will take $\frac{\partial}{\partial x^\mu}=\frac{d}{dx^{\mu}}$ throughout).

\begin{align}
\frac{\partial A^3}{\partial x}&=\left[ik\ai + ik\ao \exp(-i\Omega t)\right]\exp(i\omega t + ikx)\\
\frac{\partial^2 A^3}{\partial x^2}&=-k^2\left[\ai+\ao\exp(-i\Omega t)\right]\exp(i\omega t + ikx) \label{d2Adx2}\\
\frac{\partial A^3}{\partial t}&=\left[-\omega\exp(-i\omega t) -i(\omega + \Omega)\ao\exp(-i(\omega+\Omega)t)
+\frac{d\ao}{dt}\exp(-i(\omega+\Omega)t)\right]\exp(ikx) \\
\frac{\partial ^2A^3}{\partial x^2}&=\left[-\omega^2\ai-\left((\omega+\Omega)^2\ao+2i(\omega+\Omega)
\frac{d\ao}{dt}\right.\right.\nonumber\\
&\qquad \left.\left. -\cancel{\frac{d^2\ao}{dt^2}}\exp(-i(\omega+\Omega)t)\right)\exp(-i\Omega t)\right]
\exp(-i\omega t + ikx)\label{d2Adt2}
\end{align}
Since $\ao$ varies slowly we take $\frac{d^2\ao}{dt^2} \to 0$.

We can rewrite~(\ref{eq12}) as
\begin{equation}
\frac{d^2 A^3}{dt^2}=\big(1+h_+\exp(-i \Omega t)\big)\frac{d^2 A^3}{dx^2} \label{dt and dx A}
\end{equation}

Substituting~(\ref{d2Adx2}) and~(\ref{d2Adt2}) into~(\ref{dt and dx A}) we have
\begin{align}
&\omega^2\ai+\left((\omega+\Omega)^2\ao+2i(\omega+\Omega)
\frac{d\ao}{dt}\right)\exp(-i\Omega t)\\
&\quad=k^2\left[1+h_+\exp(-i\Omega t)\right]\left[\ai+\ao\exp(-i\Omega t)\right]\\
&=k^2\left[\ai + \ao\exp(-i\Omega t)+h_+\ai \exp(-i\Omega t)+ h_+\ao \exp(-2i\Omega t)\right]
\end{align}
Multiplying both sides by $\exp(i\Omega t)$,
\begin{align}
\cancel{\omega^2\ai\exp(i\Omega t)}+(\omega + \Omega)^2\ao + 2i(\omega+\Omega)\frac{d\ao}{dt}
&=\cancel{k^2\ai\exp(i\Omega t)}+k^2\ao + h_+ k^2\ai + k^2 h_+\ao \exp(-i\Omega t)
\end{align}
where we recall $k=\omega$ to cancel some terms above. We shall interchange $k$ and $\omega$ in what follows where convenient.

\begin{align}
h_+ k^2\ai&=
(\omega+\Omega)^2\ao - k^2\ao - k^2h_+\ao \exp(-i\Omega t)+2i(\omega + \Omega)\frac{d\ao}{dt}\\
&=\omega^2\underbrace{\left(1+\frac{\Omega}{\omega}\right)^2}_{\mathclap{\approx 1+\frac{2\Omega}{\omega}}}\ao-\omega^2 \ao -k^2h_+ \ao \exp(-i\Omega t)+2i\omega\underbrace{\left(1+\frac{\Omega}{\omega}\right)}_{\approx 1}\frac{d\ao}{dt}\\
&=\cancel{\omega^2\ao}+2\omega\Omega \ao -\cancel{k^2\ao}- k^2 h_+\ao\exp(-i\Omega t) 
+ 2i\omega\frac{d\ao}{dt}
\end{align}
Multiplying both sides by $-i$,
\begin{align}
-h_+ k^2 \ai&=2\omega \frac{d\ao}{dt}-2i\omega\Omega \ao + ik^2 h_+ \ao \exp(i\Omega t)
\intertext{Now dividing both sides by $2\omega$,}
-\frac{ih_+k^2}{2\omega}\ai&=\frac{d\ao}{dt}-i\Omega \ao+\frac{ih_+ k^2}{2\omega}\ao\exp(-i\Omega t)\\
\Rightarrow -\frac{ih_+k}{2}\ai&=\frac{d\ao}{dt}-i\Omega \ao+\frac{ih_+ k}{2}\ao\exp(-i\Omega t)
\end{align}
Finally, taking the $\frac{ih_+ k}{2}\ao\exp(-i\Omega t)$ term as negligible, we conclude
\begin{equation}
\frac{dA_{\text{out}}}{dt} - i\Omega A_{\text{out}} = -\frac{i h_+ k A_{\text{in}}}{2}
\end{equation}
as required!



\begin{framed}
\textbf{iv.} Equation~(\ref{eq14}) has the simple solution (you don't need to prove this)
\begin{equation}
A_{\text{out}}(t) = -\frac{h_+ k A_{\text{in}}}{2\Omega}[\exp(i\Omega t) - 1]\label{eq15}
\end{equation}
for the initial condition $A_{\text{out}}=0$ at $t=0$. By writing out the laser field after one round-trip time in the approximate form $A^3 = A_{\text{in}}\exp(-i\omega t + ikx - i\phi)$, show that~(\ref{eq15}) introduces a phase correction
\begin{equation}
\phi = \frac{h_+ \omega}{\Omega}\sin(\Omega L) \exp(-i \Omega L) \label{eq16}
\end{equation}
The result~(\ref{eq16}) applies to one arm of the interferometer ($x$-axis). The phase correction in the second arm is equal but opposite. Hence the total phase difference between the arms is twice~(\ref{eq16}), which is exactly the same as the ray optics result~(\ref{eq5})!
\end{framed}

Starting from~(\ref{eq13}) then substituting in~(\ref{eq15}), we have
\begin{align}
A^3 &=\ai\exp(-i\omega t + ikx)+\ao\exp(-i\Omega t)\exp(-i\omega t + ikx)\\
&=\ai\exp(-i\omega t + ikx)\left[1-\frac{h_+k}{2\Omega}\left(\exp(i\Omega t)-1\right)\exp(-i\Omega t)\right]\\
&=\ai\exp(-i\omega t + ikx)\left[1-\frac{h_+k}{2\Omega}\left(1-\exp(-i\Omega t)\right)\right]
\end{align}

Comparing this with the approximate form of $A^3$ in terms of phase shift $\phi$,
\begin{equation}
A^3 = \ai\exp(-i\omega t + ikx -i\phi)
\end{equation}
we have
\begin{equation}
\exp(-i\phi)=1-\frac{h_+k}{2\Omega}\left[1-\exp(-i\Omega t)\right]
\end{equation}

For small $\phi$, we see
\begin{equation}
\exp(-i\phi)=\cos(\phi)-i\sin(\phi)\approx 1-i\phi
\end{equation}
Hence we can find $\phi$ as
\begin{align}
1-i\phi &= 1-\frac{h_+k}{2\Omega}\left[1-\exp(-i\Omega t)\right]\\
\Rightarrow \phi &= \frac{h_+k}{2}\frac{1}{2i}\left[\exp\left(\frac{i\Omega t}{2}\right)
-\exp\left(-\frac{i\Omega t}{2}\right)\right]\exp\left(-\frac{i\Omega t}{2}\right)\\
&=\frac{h_+k}{\Omega}\sin\left(\frac{\Omega t}{2}\right)\exp\left(-\frac{i\Omega t}{2}\right)\\
&=\frac{h_+\omega}{\Omega}\sin\left(\frac{\Omega t}{2}\right)\exp\left(-\frac{i\Omega t}{2}\right)
\end{align}

As we argued in (a) iv., we have $t\approx 2L$ so that we calculate the phase correction,
\begin{equation}
\phi = \frac{h_+ \omega}{\Omega}\sin(\Omega L) \exp(-i \Omega L)
\end{equation}
as required!

\end{document}


