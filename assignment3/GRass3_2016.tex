\documentclass[a4paper]{article} % A4 paper and 11pt font size

\usepackage{braket}
\usepackage{amsmath}
\usepackage{amssymb}
\usepackage{bm}
\usepackage[utf8]{inputenc}
\usepackage{verbatim}
\usepackage{tikz}
%\usepackage{tikz-feynman}
\usepackage{pgfplots}
\usepackage{pgffor}
\usepackage[version-1-compatibility]{siunitx}
\usepackage{fancyhdr}
\usepackage{mathtools} %for underbraces without whitespace (\mathclap{})


\usepackage{hyperref}
\usepackage{geometry}

 \geometry{
 a4paper,
 total={210mm,297mm},
 left=28mm,
 right=28mm,
 top=30mm,
 bottom=40mm,
 }


\usepackage{framed}
\usepackage{amssymb} %for Lagrangian L, order O
\usepackage{cancel} %for strikethroughs
\usepackage{slashed} %for Feynman slashes

\newcommand{\pmx}[1]{\begin{pmatrix}#1\end{pmatrix}}
\newcommand{\ph}[1]{\phantom{#1}}
\newcommand{\diag}{\text{diag}~}

\usepackage{gensymb}

\usepackage{fancyhdr}
\usepackage{pdflscape}
\usepackage{bm}

%for side-by-side figures
\usepackage{graphicx}
\usepackage{caption}
\usepackage{subcaption}

\setlength{\parindent}{2em}
\setlength{\parskip}{1em}
\renewcommand{\baselinestretch}{1.1}


%----------------------------------------------------------------------------------------
%	TITLE SECTION
%----------------------------------------------------------------------------------------
\setlength\parindent{0pt} % Removes all indentation from paragraphs - comment this line for an assignment with lots of text


\pagenumbering{arabic}
\begin{document}
\pagestyle{empty}

\newcommand{\HRule}{\rule{\linewidth}{0.5mm}}

\begin{titlepage}

    \begin{center}
        \textsc{\large SN: 587623}\\[6cm]

        \HRule \\[0.5cm]
		\Huge \textbf{PHYC90012 General Relativity}\\[0.5cm]
        \huge \textbf{Assignment 3}\\[0.5cm] 
        \HRule \\[1.5cm]
        \begin{minipage}{0.4\textwidth}
        \begin{center}

        \large By \\[0.75cm]
        \huge Braden \scshape Moore \\[0.5cm]
        \normalsize \normalfont Master of Science \\
        The University of Melbourne \\

        \end{center}
        \end{minipage}

        \vfill

        \large \today
    \end{center}

\newpage
\end{titlepage}
%----------------------------------------------------------------------------------------
\pagestyle{fancy}
\pagenumbering{arabic}
\rfoot{\textsc{Braden Moore, 587623}}
\lfoot{\textsc{\today}}
\rhead{\textsc{General Relativity: Assignment 3}}
\setcounter{page}{1}
\setcounter{section}{2}
\section{Playing tennis in a four-dimensional ``brane world''}

\begin{framed}
Projectile motion in a constant gravitational field is a classic introductory problem in Newtonian mechanics. One way to make a uniform gravitational field is to assemble an infinite, flat sheet of matter. If the mass per unit area in the $x-y$ plane is $\sigma$, then the Newtonian gravitational potential is given by 
\begin{equation}
\Phi_{\text{Newton}}=2\pi G\sigma |z|,\label{eq1}
\end{equation}
where $z$ is the Cartesian coordinate normal to the sheet. [Convince yourself that~(\ref{eq1}) is true by analogy with electrostatics or otherwise.] Below we ask whether it is possible to recreate this system in the strong-gravity regime in general relativity.
\end{framed}

\begin{framed}
\textbf{(a)} By appealing to symmetry, argue that the most general metric of a plane-parallel, four-dimensional spacetime takes the form 
\begin{equation}
ds^2=-e^{2\Phi (z)}dt^2 + e^{2\Psi (z)}(dx^2+dy^2)+dz^2,\label{eq2}
\end{equation}
where $\Phi(t,z)$ (not necessarily the same as $\Phi_{\text{Newton}}$), $\Psi(t,z)$, and $\Lambda(t,z)$ are functions determined by Einstein's field equations.
determined by Einstein’s field equations.
\end{framed}

The plane is symmetric under $x$ and $y \Rightarrow g_{xx}=g_{yy}$. By symmetry, the spacetime is unchanged under $x\to -x$, $y\to -y$, $z\to -z$, $t\to -t$ transformations $\Rightarrow$ cross-terms are zero, i.e. $g_{xy}=-g_{xy}=0$.

Recall that the spacetime interval $ds^2$ can be written as,
\begin{equation}
ds^2=g_{00}dt^2+g_{11}dx^2+g_{22}dy^+g_{33}dz^2
\end{equation}
so that we have
\begin{equation}
ds^2=-e^{2\Phi(z,t)}dt^2+e^{2\Psi(z,t)}(dx^2+dy^2)+e^{2\Lambda (z,t)}dz^2
\end{equation}
where $g_{00}<0$ and $g_{11},g_{22}>0$ everywhere.

We have introduced functions $\Phi, \Psi, \Lambda$ which depend on $z,t$ by the geometry given.

\begin{framed}
\textbf{(b) i.} If the spacetime is static (caution: such a solution may not necessarily exist), construct explicitly a coordinate transformation, that puts~(\ref{eq2}) into the form 
\begin{equation}
ds^2=-e^{2\Phi(z)} dt^2+e^{2\Psi(z)}(dx^2+dy^2)+dz^2,\label{eq3}
\end{equation}
where $t$, $x$, $y$, and $z$ are suitably relabelled from~(\ref{eq2}).
\end{framed}

For a static spacetime (no time dependence), we can rewrite the functions independent of time, i.e.
\begin{align*}
e^{\Phi(z)}dt'&=e^{\Phi(z,t)}dt\\
e^{\Psi(z)}dx'&=e^{\Psi(z,t)}dx\\
e^{\Psi(z)}dy'&=e^{\Psi(z,t)}dy\\
dx'&=e^{\Lambda(z,t)}dz
\end{align*}

Assuming the functions $\Phi,\Psi$ can be re-expressed in a time independent way, we will only need the transformation $dz'=e^{\Lambda(z,t)}dz$

This leaves us with
\begin{equation}
ds^2=-e^{2\Phi(z)} dt^2+e^{2\Psi(z)}(dx^2+dy^2)+dz^2
\end{equation}
as required.



\begin{framed}
\textbf{(b) ii.} Explain why it is impossible to get rid of the $e^{2\Lambda(r)}$ factor multiplying $dr^2$ in the ``standard'' Schwarzschild metric in the same way, such that the metric contains only one undetermined function $e^{2\Phi(r)}$.
\end{framed}

Using Schwarzchild metric $dr'=\left(1-\frac{2M}{r}\right)^{-1/2}dr$ we find
\begin{equation}
r'=\sqrt{r^2-2Mr}+M\log\left(\sqrt{r^2-2Mr}+r-m\right)
\end{equation}

At $r=0$ the second term becomes $M\log(-m)$, which is undefined since $m>0$.

{\huge Wrong?}

\begin{framed}
\textbf{(c)} Nine of the Christoffel symbols associated with~(\ref{eq3}) are nonzero. Calculate them. Here and henceforth, please feel free to use a symbolic algebra package like \emph{Mathematica} to make your life easier.
\end{framed}

These Christoffel symbols have been calculated used Mathematica (see attached code).

\begin{align}
\Gamma^{1}_{14}&=\Gamma^{1}_{41}=\Phi'(z)\\
\Gamma^{2}_{42}&=\Gamma^{2}_{24}=\Psi'(z)\\
\Gamma^{3}_{43}&=\Gamma^{3}_{34}=\Psi'(z)\\
\Gamma^{4}_{11}&=e^{2\Phi}\Phi'\\
\Gamma^{4}_{22}&=-e^{2\Psi}\Psi'\\
\Gamma^{4}_{33}&=-e^{2\Psi}\Psi'
\end{align}

\begin{framed}
\textbf{(d)} Show that the nonzero components of the Ricci tensor $R_{\mu\nu}$ are
\begin{align}
R_{tt}&=e^{2\Phi}[(\Phi')^2+2\Phi' \Psi' +\Phi''],\\
R_{xx}&=-e^{2\Phi}[2(\Psi')^2+\Psi' \Phi' + \Psi'']\\
R_{yy}&=R_{xx},\\
R_{zz}&=-(\Phi')^2-2(\Psi')^2-\Phi''-2\Psi''.
\end{align}
Primes denote derivatives with respect to $z$.
\end{framed}

We recall the Ricci tensor $R_{\alpha\beta}$ is a contraction on the first and third indices of the Riemann tensor, written as
\begin{equation}
R_{\alpha\beta}=R^{i}_{\ph{i}\alpha i \beta}
\end{equation}
We also recall that the Riemann tensor in the form $R^{\alpha}_{\beta\mu\nu}$ can be written as
\begin{equation}
R^{\rho}_{\ph{\rho}\sigma \mu\nu}=\partial_{\mu}\Gamma^{\rho}_{\ph{\rho}\nu\sigma}
-\partial_{\nu}\Gamma^{\rho}_{\ph{\rho}\mu\sigma}
+\Gamma^{\rho}_{\ph{\rho}\mu\lambda}\Gamma^{\lambda}_{\ph{\lambda}\nu \sigma}
-\Gamma^{\rho}_{\ph{\rho}\nu\lambda}\Gamma^{\lambda}_{\ph{\lambda}\mu\sigma}
\end{equation}

The non-zero components of the Ricci tensor are calculated in Mathematica as shown in the code, and we find

\begin{align}
R_{tt}&=e^{2\Phi}[(\Phi')^2+2\Phi' \Psi' +\Phi''],\\
R_{xx}&=-e^{2\Phi}[2(\Psi')^2+\Psi' \Phi' + \Psi'']\\
R_{yy}&=R_{xx},\\
R_{zz}&=-(\Phi')^2-2(\Psi')^2-\Phi''-2\Psi'',
\end{align}

as required.


\begin{framed}
\textbf{(e)} Show that the nonzero contravariant components of Einstein’s field equations with
cosmological constant $\Lambda$ and stress-energy tensor $T_{\mu\nu}$ are
\begin{align}
8\pi T^{tt}&=-e^{-2\Phi}[3(\Psi')^2+2\Psi''+\Lambda],\\
8\pi T^{xx}&=e^{-2\Psi}[(\Phi')^2+\Phi' \Psi' + (\Psi')^2+\Phi'' + \Psi'' + \Lambda],\\
8\pi T^{yy}&=8\pi T^{xx},\\
8\pi T^{zz}&=\Psi'(2\Phi' + \Psi')+\Lambda.
\end{align}
You can do this with pen and paper but will find \emph{Mathematica} more soothing.
\end{framed}

We know
\begin{align}
8\pi T_{\mu\nu}&=R_{\mu\nu}-\frac{1}{2}g_{\mu\nu}R+g_{\mu\nu}\Lambda\\
\Rightarrow 8\pi T^{\alpha\beta}&=g^{\alpha\mu}g^{\beta\nu}8\pi T_{\mu\nu}
\end{align}

The values of $8\pi T^{\alpha\alpha}$ are calculated as shown in the code, and we find

\begin{align}
8\pi T^{tt}&=-e^{-2\Phi}[3(\Psi')^2+2\Psi''+\Lambda],\\
8\pi T^{xx}&=e^{-2\Psi}[(\Phi')^2+\Phi' \Psi' + (\Psi')^2+\Phi'' + \Psi'' + \Lambda],\\
8\pi T^{yy}&=8\pi T^{xx},\\
8\pi T^{zz}&=\Psi'(2\Phi' + \Psi')+\Lambda,
\end{align}

as required.



\begin{framed}
\textbf{(f)} Einstein’s field equations place strong constraints on the physical form of the
stress-energy consistent with the metric~(\ref{eq3}).
\end{framed}

\begin{framed}
\textbf{i.} Explain why it is impossible to generate~(\ref{eq3}) with a sheet of cold matter (dust).
\end{framed}

Starting with
\begin{equation}
8\pi T^{tt}=-e^{-2\Phi}\left(3(\Psi')^2+2\Psi''+\Lambda\right)
\end{equation}
we let $T^{tt}=\delta(z)$ as we are confied to the sheet $z=0$. As the LHS has a delta-function, the right hand side must also be proportional to a delta-function.

As a first derivative cannot be a delta-function (proof not shown), as as $\Lambda=0$, we see
\begin{equation}
\Phi''\sim \delta(z)
\end{equation}

We also have
\begin{align}
8\pi T^{xx}&=0=\Phi'^2+\Phi'\Psi'+\Psi'^2+\Phi'' + \Psi'' + \Lambda \label{xx 1}\\
8\pi T^{zz}&=0=\Psi'(2\Phi' + \Psi') + \Lambda \label{zz 1}
\end{align}
Subtracting~(\ref{zz 1}) from~(\ref{xx 1}) we find
\begin{equation}
\Phi'' = -\Psi''
\end{equation}
which implies that $\Phi''$ must be a delta-function.

Similarly, subtracting~(\ref{xx 1}) from~(\ref{zz 1}) we find
\begin{align}
0&=(\Phi')^2-\Phi' \Psi'\\
\therefore \Psi'&=0 \text{ or } \Phi'=\Psi'
\end{align}

However, both of these possibilities contradict with our previous findings! Hence we cannot generate~(\ref{eq3}) with a sheet of cold matter.


\begin{framed}
\textbf{ii.} By thinking about a suitable initial value problem qualitatively, speculate why this seemingly natural system - a static dust sheet - is impossible to assemble ``from the ground up'' in general relativity, even though there is no obstacle in a Newtonian context. What might ``go wrong''? Do not try calculating anything, unless you are in the mood to win a Nobel Prize.
\end{framed}

If we are to build a sheet from the ground up, i.e. particle by particle, we would find that as we add a particle, a gravitational or electric force would act on the other particles currently in the ``sheet'' and distory the structure. Hence, a flat sheet structure would be impossible to form.


\begin{framed}
\textbf{(g)} Consider the special case $\Psi=0$, i.e. warped time, Euclidean space, no preferential
warping of space in the $z$ direction relative to the $x$ and $y$ directions.
\end{framed}

\begin{framed}
\textbf{i.} In the bulk ($z=0$), where there is zero stress-energy, show that $\Lambda=0$ must
hold to obtain a self-consistent solution of the form~(\ref{eq3}).
\end{framed}

Zero stree energy leads to
\begin{equation}
8\pi T^{tt}=0=\Psi'(2\Phi' + \Psi) + \Lambda = \Lambda
\end{equation}
since $\Psi=0\Rightarrow \Psi'=0$.
\begin{equation}
\therefore \Lambda = 0
\end{equation}


\begin{framed}
\textbf{ii.} Hence show that the stress-energy must vanish on the sheet $z = 0$ too.
\end{framed}

At $z=0$, there will be no mass as $\Lambda=0,\Psi=0$
\begin{itemize}
\item[] $\Rightarrow$ no invariant mass $E_{m}$
\item[] $\Rightarrow$ no energy density
\item[] $\therefore T^{tt}=0$
\item[] $\therefore$ momentum flow must also be zero
\end{itemize}

{\huge Wrong?}

\begin{framed}
\textbf{iii.} Prove that~(\ref{eq3}) reduces to the Rindler metric for a uniformly accelerated frame.
What is the implied acceleration?

In the special case $\Phi = 0$, a similar bulk-then-sheet approach yields the Minkowski
metric.
\end{framed}

\begin{equation}
ds^2=-e^{2\Phi(z)}dt^2 + dx^2 + dy^2 + dz^2
\end{equation}

If $z$ is small we can Taylor expand
\begin{equation}
\therefore ds^2 = -\left(1+2\Phi'(z)\right)dt^2 + dx^2 + dy^2 + dz^2
\end{equation}

This is of the form of an accelerating frame with acceleration $2\Phi'(z)$.

\begin{framed}
\textbf{(h)} Consider the special case $\Psi=\Phi$. This corresponds to preferentially warping space in $z$ relative to $x$ and $y$, i.e. the ``symmetric'' coordinates $(t, x, y)$ are treated as a Minkowski subspace with warping in the $z$ coordinate. The system is the four dimensional analogue of the five-dimensional \emph{Randall-Sundrum 1-brane} spacetime, which revolutionized the study of brane worlds 
\end{framed}


\begin{framed}
\textbf{i.} In the bulk, show that one obtains
\begin{equation}
\Phi=g |z|,\label{eq12}
\end{equation}
if $\Phi(z)$ vanishes at $z = 0$ without loss of generality. In~(\ref{eq12}), $g$ is an integration
constant with units of acceleration.
\end{framed}


\begin{align}
\Phi = \Psi \Rightarrow 0 &= 3(\Phi') + \Lambda\\
\therefore \frac{d\Phi}{dz}&= \pm \sqrt{-\frac{\Lambda}{3}}\\
\therefore \Phi &= \pm \sqrt{-\frac{\Lambda}{3}}|z|,\quad\text{as z is symmetric}\\
&=g|z|
\end{align}


\begin{framed}
\textbf{ii.} Prove that $g$ satisfies the fine-tuning condition
\begin{equation}
\Lambda=-3g^2.\label{eq13}
\end{equation}
Comment on the physical significance of the sign of $\Lambda$.
\end{framed}

\begin{equation}
g=\pm \sqrt{-\frac{\Lambda}{3}} \therefore \Lambda = -3g^2
\end{equation}
If $\Lambda$ is positive $\Rightarrow$ imaginary $g$ which is not physical. $\Rightarrow \Lambda$ is negative.

{\huge Wrong?}

\begin{framed}
\textbf{iii.} Assume that the stress-energy is confined to the sheet, i.e. $T^{\mu\nu}\propto \delta(z)$. Prove that the stress-energy tensor takes the physically unusual form
\begin{equation}
T^{\mu\nu}=(2\pi)^{-1}g\delta(z)\diag(-1,1,1,0).\label{eq14}
\end{equation}
Equation~(\ref{eq14}) implies fine tuning between $T^{\mu\nu}$ and $\Lambda$ via $g$.
\end{framed}

\begin{equation}
\Phi = g|z| \therefore \Phi''= 2g\delta(z)
\end{equation}
as 
\begin{align}
\frac{d\Phi}{dz}&=\text{sgn}(z)\\
\frac{d}{dz}\text{sgn}(z)&=z\delta(z)
\end{align}

Also, 
\begin{align}
3(\Phi')^2&=-\Lambda\\
\therefore 8\pi T^{tt}&=-e^{2g|z|}(4\Phi'')\\
\therefore T{tt}&= -e^{2g|z|} (2\pi)^{-1} \delta(z)\\
8\pi T^{xx}&= e^{2g|z|}(4\Phi'')\\
T^{xx}&=(2\pi)^{-1}e^{2g|z|} \delta(z)\\
T^{yy}&=T^{xx}\\
8\pi T^{zz}&=0\\
\Rightarrow T^{zz}&=0
\end{align}
which leaves us with
\begin{equation}
T^{\mu\nu}=(2\pi)^{-1}g\delta(z)\diag(-1,1,1,0),
\end{equation}
as required.


\begin{framed}
\textbf{iv.} Interpret physically the two cases $g < 0$ and $g > 0$.
\end{framed}

$g<0$ positive energy density, normal stress (pressure) is negative.

$g>0$ negative energy density, normal stress (pressure) is positive.


\begin{framed}
\textbf{v.} Interpret physically the signs of $T^{tt}$ and $T^{xx} = T^{yy}$ for $g > 0$.
\end{framed}

For $g>0$, $T^{tt}$ is negative, $T^{xx}=T^{yy}$ is positive.

\begin{framed}
\textbf{vi.}  Explain physically why $T^{zz} = 0$ makes sense in terms of a plausible microscopic
(``particulate'') model of the sheet.
\end{framed}

$T^{zz}$ is pressure in the $z-$direction. But if there was pressure, this leads to motion in $z$ of the sheet, and the sheet would buckle. But since our sheet is flat, there must not be any pressure in $z \Rightarrow T^{zz}=0$.

\begin{framed}
\textbf{vii.} Prove that it is impossible to build the sheet out of a perfect fluid or dust.
\end{framed}

For Schutz Ch. 4, for a perfect fluid the stress-energy tensor is given by
\begin{equation}
T^{\mu\nu}=\text{diag}(\rho,p,p,p)\label{perfect fluid}
\end{equation}

But our $T^{\mu\nu}$ is confined to the sheet
\begin{equation}
T^{\mu\nu}\propto \text{diag}(-1,1,1,0)\label{sheet}
\end{equation}

In the perfect fluid~(\ref{perfect fluid}), we see the last two elements of the tensor are identical. However, on the sheet~(\ref{sheet}) we see the last two elements are different $(1\neq 0) \Rightarrow $ we cannot build a sheet out of a perfect fluid!

\begin{framed}
\textbf{(i)} A tennis player standing on the sheet hits a tennis ball into the $z>0$ half-space with initial 4-velocity $u(0)$.
\end{framed}

\begin{framed}
\textbf{i.} Starting from the geodesic equations in the form
\begin{equation}
\frac{du_{\nu}}{d\tau}=\frac{1}{2}g_{\alpha\beta,\nu}u^{\alpha}u^{\beta},\label{eq15}
\end{equation}
identify three constants of the motion. In~(\ref{eq15}), $\tau$ denotes the proper time as measured in the ball's momentarily comoving reference frame.
\end{framed}

\begin{framed}
\textbf{ii.} From the z component of~(\ref{eq15}), show that
\begin{equation}
\frac{du^z}{d\tau}=-g[1+(u^z)^2].\label{eq16}
\end{equation}
What is unusual physically about this acceleration?
\end{framed}

\begin{framed}
\textbf{iii.} Similarly or otherwise, prove the equivalent result
\begin{equation}
\left(\frac{dz}{d\tau}\right)^2=-1+e^{-2gz}\{1+[u^z(0)]^2\}.\label{eq17}
\end{equation}
\end{framed}

\begin{framed}
\textbf{iv.} Solve either~(\ref{eq16}) or~(\ref{eq17}) to obtain
\begin{equation}
z=g^{-1}\ln[\cos g\tau+u^z(0)\sin g\tau].\label{eq18}
\end{equation}
Please note that there are many valid ways to prove~(\ref{eq16})-(\ref{eq18}).
\end{framed}

\begin{framed}
\textbf{(j)} Show that the ball reaches the apex of its trajectory at proper time
\begin{equation}
\tau_{\text{top}}=g^{-1}\tan^{-1}u^z(0),
\end{equation}
and that the associated coordinate time is
\begin{equation}
t_{\text{top}}=\frac{u^z(0)\{1+[u^x(0)]^2+[u^z(0)]^2\}^{1/2}}{g\{1+[u^z(0)]^2\}},
\end{equation}
in the coordinates in which~(\ref{eq3}) is written. Interestingly, $\tau_{\text{top}}$ is finite, no matter
how hard the ball is struck initially. Does the same hold true for $t_{\text{top}}$?
\end{framed}

\begin{framed}
\textbf{(k)} Show that the horizontal distance traversed by the ball from its launch point to where it lands on the sheet, as measured in the coordinates associated with~(\ref{eq3}), is given by
\begin{equation}
x_{\text{land}}=\frac{2u^x(0)u^z(0)}{g\{1+[u^z(0)]^2\}}.\label{eq21}
\end{equation}
How does~(\ref{eq21}) differ qualitatively from Newtonian projectile motion? Should we expect $x_{\text{land}}$ to equal the $x$-component of the initial 3-velocity multiplied by $2t_{\text{top}}$,
since the metric does not depend on $x$?
\end{framed}

\begin{framed}
\textbf{(l)} Finally, suppose that instead of a tennis ball we launch a laser beam into the $z > 0$ half-space with initial 4-momentum $p(0)$ per photon.
\end{framed}

\begin{framed}
\textbf{i.} Let $\lambda$ be an affine parameter tracing out the light ray. Starting from the geodesic equations or otherwise, solve for the photon’s trajectory. You should find
\begin{equation}
z=g^{-1}\ln[1+p^z(0)g\lambda].
\end{equation}
In other words, the photon never falls back to the sheet, no matter how much stress-energy the sheet contains.
\end{framed}

\begin{framed}
\textbf{ii.} Consider an intergalactic variant of the Pound-Rebka experiment, in which we fly horizontally (in the $x$ direction, say) in a rocket at constant speed $V$ [as measured in the coordinates associated with~(\ref{eq3})] while maintaining a constant altitude $z_{\text{rocket}}$ above the sheet. Show that the laser frequency measured by the stationary emitter $\nu_{\text{em}}$, and the frequency measured by the experimentalist in the rocket, $\nu_{\text{rec}}$, are related according to
\begin{equation}
\frac{\nu_{\text{rec}}}{\nu_{\text{em}}}=\frac{e^{-gz_{\text{rocket}}}}{(1-V^2)^{1/2}}
\left[1-\frac{Vp^x(0)}{\{[p^x(0)]^2+[p^z(0)]^2\}^{1/2}}\right].
\end{equation}

Most of the physics in this question applies in five dimensions too. It is interesting to think about it in the context of Randall-Sundrum brane worlds.
\end{framed}


\end{document}







































