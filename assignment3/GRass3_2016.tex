\documentclass[a4paper]{article} % A4 paper and 11pt font size

\usepackage{braket}
\usepackage{amsmath}
\usepackage{amssymb}
\usepackage{bm}
\usepackage[utf8]{inputenc}
\usepackage{verbatim}
\usepackage{tikz}
%\usepackage{tikz-feynman}
\usepackage{pgfplots}
\usepackage{pgffor}
\usepackage[version-1-compatibility]{siunitx}
\usepackage{fancyhdr}
\usepackage{mathtools} %for underbraces without whitespace (\mathclap{})


\usepackage{hyperref}
\usepackage{geometry}

 \geometry{
 a4paper,
 total={210mm,297mm},
 left=28mm,
 right=28mm,
 top=30mm,
 bottom=40mm,
 }


\usepackage{framed}
\usepackage{amssymb} %for Lagrangian L, order O
\usepackage{cancel} %for strikethroughs
\usepackage{slashed} %for Feynman slashes

\newcommand{\pmx}[1]{\begin{pmatrix}#1\end{pmatrix}}
\newcommand{\ph}[1]{\phantom{#1}}
\newcommand{\diag}{\text{diag}}

\usepackage{gensymb}

\usepackage{fancyhdr}
\usepackage{pdflscape}
\usepackage{bm}

%for side-by-side figures
\usepackage{graphicx}
\usepackage{caption}
\usepackage{subcaption}

\setlength{\parindent}{2em}
\setlength{\parskip}{1em}
\renewcommand{\baselinestretch}{1.1}


%----------------------------------------------------------------------------------------
%	TITLE SECTION
%----------------------------------------------------------------------------------------
\setlength\parindent{0pt} % Removes all indentation from paragraphs - comment this line for an assignment with lots of text


\pagenumbering{arabic}
\begin{document}
\pagestyle{empty}

\newcommand{\HRule}{\rule{\linewidth}{0.5mm}}

\begin{titlepage}

    \begin{center}
        \textsc{\large SN: 587623}\\[6cm]

        \HRule \\[0.5cm]
		\Huge \textbf{PHYC90012 General Relativity}\\[0.5cm]
        \huge \textbf{Assignment 3}\\[0.5cm] 
        \HRule \\[1.5cm]
        \begin{minipage}{0.4\textwidth}
        \begin{center}

        \large By \\[0.75cm]
        \huge Braden \scshape Moore \\[0.5cm]
        \normalsize \normalfont Master of Science \\
        The University of Melbourne \\

        \end{center}
        \end{minipage}

        \vfill

        \large \today
    \end{center}

\newpage
\end{titlepage}
%----------------------------------------------------------------------------------------
\pagestyle{fancy}
\pagenumbering{arabic}
\rfoot{\textsc{Braden Moore, 587623}}
\lfoot{\textsc{\today}}
\rhead{\textsc{General Relativity: Assignment 3}}
\setcounter{page}{1}
\setcounter{section}{2}
\section{Playing tennis in a four-dimensional ``brane world''}

\begin{framed}
Projectile motion in a constant gravitational field is a classic introductory problem in Newtonian mechanics. One way to make a uniform gravitational field is to assemble an infinite, flat sheet of matter. If the mass per unit area in the $x-y$ plane is $\sigma$, then the Newtonian gravitational potential is given by 
\begin{equation}
\Phi_{\text{Newton}}=2\pi G\sigma |z|,\label{eq1}
\end{equation}
where $z$ is the Cartesian coordinate normal to the sheet. [Convince yourself that~(\ref{eq1}) is true by analogy with electrostatics or otherwise.] Below we ask whether it is possible to recreate this system in the strong-gravity regime in general relativity.
\end{framed}

\begin{framed}
\textbf{(a)} By appealing to symmetry, argue that the most general metric of a plane-parallel, four-dimensional spacetime takes the form 
\begin{equation}
ds^2=-e^{2\Phi (z)}dt^2 + e^{2\Psi (z)}(dx^2+dy^2)+dz^2,\label{eq2}
\end{equation}
where $\Phi(t,z)$ (not necessarily the same as $\Phi_{\text{Newton}}$), $\Psi(t,z)$, and $\Lambda(t,z)$ are functions determined by Einstein's field equations.
determined by Einstein’s field equations.
\end{framed}

An infinite flat sheet in the $x-y$ plane is is symmetric under interchange of $x$ and $y$, so $g_{xx}=g_{yy}$. By symmetry of the sheet, under the transformations $x\to -x$, $y\to -y$, $z\to -z$, $t\to -t$ the spacetime is unchanged $\Rightarrow g_{\alpha\beta}=-g_{\beta\alpha}=0$ (for $\alpha\neq \beta$).

Recall that the spacetime interval $ds^2$ can be written as
\begin{equation}
ds^2=g_{00}dt^2+g_{11}dx^2+g_{22}dy^{2}+g_{33}dz^2
\end{equation}


so that we have
\begin{equation}
ds^2=-e^{2\Phi(z,t)}dt^2+e^{2\Psi(z,t)}(dx^2+dy^2)+e^{2\Lambda (z,t)}dz^2
\end{equation}
where the functions $\Phi, \Psi, \Lambda$ depend on $z,t$, and $g_{11},g_{22}>0,g_{00}<0$  everywhere.


\begin{framed}
\textbf{(b) i.} If the spacetime is static (caution: such a solution may not necessarily exist), construct explicitly a coordinate transformation, that puts~(\ref{eq2}) into the form 
\begin{equation}
ds^2=-e^{2\Phi(z)} dt^2+e^{2\Psi(z)}(dx^2+dy^2)+dz^2,\label{eq3}
\end{equation}
where $t$, $x$, $y$, and $z$ are suitably relabelled from~(\ref{eq2}).
\end{framed}

A static spacetime has no time dependence, so we can rewrite the functions independent of time, i.e.
\begin{align*}
e^{\Phi(z)}dt'&=e^{\Phi(z,t)}dt\\
e^{\Psi(z)}dx'&=e^{\Psi(z,t)}dx\\
e^{\Psi(z)}dy'&=e^{\Psi(z,t)}dy\\
dx'&=e^{\Lambda(z,t)}dz
\end{align*}

Assuming the functions $\Phi,\Psi$ can be re-expressed in a time independent way, we will only need the transformation $dz'=e^{\Lambda(z,t)}dz$.

This leaves us with
\begin{equation}
ds^2=-e^{2\Phi(z)} dt^2+e^{2\Psi(z)}(dx^2+dy^2)+dz^2\label{ds2 metric}
\end{equation}
as required.



\begin{framed}
\textbf{(b) ii.} Explain why it is impossible to get rid of the $e^{2\Lambda(r)}$ factor multiplying $dr^2$ in the ``standard'' Schwarzschild metric in the same way, such that the metric contains only one undetermined function $e^{2\Phi(r)}$.
\end{framed}

Using the Schwarzchild metric $dr'=\left(1-\frac{2M}{r}\right)^{-1/2}dr$, we find
\begin{equation}
r'=\sqrt{r^2-2Mr}+M\log\left(\sqrt{r^2-2Mr}+r-m\right)
\end{equation}

At $r=0$ the second term becomes $M\log(-m)$, which is undefined since $m>0$.

\begin{framed}
\textbf{(c)} Nine of the Christoffel symbols associated with~(\ref{eq3}) are nonzero. Calculate them. Here and henceforth, please feel free to use a symbolic algebra package like \emph{Mathematica} to make your life easier.
\end{framed}

We can use the expression for the Christoffel symbols in terms of the metric,
\begin{equation}
\Gamma^{\gamma}_{\beta\mu}=\frac{1}{2}g^{\alpha\gamma}\left(
g_{\alpha\beta,\mu}+g_{\alpha\mu,\beta}-g_{\beta\mu,\alpha}\right)
\end{equation}
where we have the metric components $g_{\alpha\alpha}$ as given by~\ref{ds2 metric}) (with all non-diagonal terms zero); i.e.
\begin{equation}
g_{\alpha\beta}=\diag\left(-e^{2\Phi(z)},e^{2\Psi(z)},e^{2\Psi(z)},1\right)
\end{equation}

Now, using a mathematical calculator (specifically an \emph{TI-89 Titanium} calculator) to make our lives easier, we calculate the Christoffel symbols and find the non-zero terms as

\begin{align}
\Gamma^{1}_{14}&=\Gamma^{1}_{41}=\Phi'(z)\\
\Gamma^{2}_{42}&=\Gamma^{2}_{24}=\Psi'(z)\\
\Gamma^{3}_{43}&=\Gamma^{3}_{34}=\Psi'(z)\\
\Gamma^{4}_{11}&=e^{2\Phi}\Phi'\\
\Gamma^{4}_{22}&=-e^{2\Psi}\Psi'\\
\Gamma^{4}_{33}&=-e^{2\Psi}\Psi'
\end{align}

\begin{framed}
\textbf{(d)} Show that the nonzero components of the Ricci tensor $R_{\mu\nu}$ are
\begin{align}
R_{tt}&=e^{2\Phi}[(\Phi')^2+2\Phi' \Psi' +\Phi''],\\
R_{xx}&=-e^{2\Phi}[2(\Psi')^2+\Psi' \Phi' + \Psi'']\\
R_{yy}&=R_{xx},\\
R_{zz}&=-(\Phi')^2-2(\Psi')^2-\Phi''-2\Psi''.
\end{align}
Primes denote derivatives with respect to $z$.
\end{framed}

We recall the Ricci tensor $R_{\alpha\beta}$ is a contraction on the first and third indices of the Riemann tensor, written as
\begin{equation}
R_{\alpha\beta}=R^{i}_{\ph{i}\alpha i \beta}
\end{equation}
We also recall that the Riemann tensor in the form $R^{\alpha}_{\beta\mu\nu}$ can be written as
\begin{equation}
R^{\rho}_{\ph{\rho}\sigma \mu\nu}=\partial_{\mu}\Gamma^{\rho}_{\ph{\rho}\nu\sigma}
-\partial_{\nu}\Gamma^{\rho}_{\ph{\rho}\mu\sigma}
+\Gamma^{\rho}_{\ph{\rho}\mu\lambda}\Gamma^{\lambda}_{\ph{\lambda}\nu \sigma}
-\Gamma^{\rho}_{\ph{\rho}\nu\lambda}\Gamma^{\lambda}_{\ph{\lambda}\mu\sigma}
\end{equation}

We calculate the components of the Ricci tensor using our calculator, and find the non-zero components to be

\begin{align}
R_{tt}&=e^{2\Phi}[(\Phi')^2+2\Phi' \Psi' +\Phi''],\\
R_{xx}&=-e^{2\Phi}[2(\Psi')^2+\Psi' \Phi' + \Psi'']\\
R_{yy}&=R_{xx},\\
R_{zz}&=-(\Phi')^2-2(\Psi')^2-\Phi''-2\Psi'',
\end{align}

as required.


\begin{framed}
\textbf{(e)} Show that the nonzero contravariant components of Einstein’s field equations with
cosmological constant $\Lambda$ and stress-energy tensor $T_{\mu\nu}$ are
\begin{align}
8\pi T^{tt}&=-e^{-2\Phi}[3(\Psi')^2+2\Psi''+\Lambda],\\
8\pi T^{xx}&=e^{-2\Psi}[(\Phi')^2+\Phi' \Psi' + (\Psi')^2+\Phi'' + \Psi'' + \Lambda],\\
8\pi T^{yy}&=8\pi T^{xx},\\
8\pi T^{zz}&=\Psi'(2\Phi' + \Psi')+\Lambda.
\end{align}
You can do this with pen and paper but will find \emph{Mathematica} more soothing.
\end{framed}

We know
\begin{align}
8\pi T_{\mu\nu}&=R_{\mu\nu}-\frac{1}{2}g_{\mu\nu}R+g_{\mu\nu}\Lambda\\
\Rightarrow 8\pi T^{\alpha\beta}&=g^{\alpha\mu}g^{\beta\nu}8\pi T_{\mu\nu}
\end{align}

From this, we can calculate the values of $8\pi T^{\alpha\alpha}$ using our calculator, with the results

\begin{align}
8\pi T^{tt}&=-e^{-2\Phi}[3(\Psi')^2+2\Psi''+\Lambda],\\
8\pi T^{xx}&=e^{-2\Psi}[(\Phi')^2+\Phi' \Psi' + (\Psi')^2+\Phi'' + \Psi'' + \Lambda],\\
8\pi T^{yy}&=8\pi T^{xx},\\
8\pi T^{zz}&=\Psi'(2\Phi' + \Psi')+\Lambda,
\end{align}

as required.



\begin{framed}
\textbf{(f)} Einstein’s field equations place strong constraints on the physical form of the
stress-energy consistent with the metric~(\ref{eq3}).
\end{framed}

\begin{framed}
\textbf{i.} Explain why it is impossible to generate~(\ref{eq3}) with a sheet of cold matter (dust).
\end{framed}

Starting with
\begin{equation}
8\pi T^{tt}=-e^{-2\Phi}\left(3(\Psi')^2+2\Psi''+\Lambda\right)
\end{equation}
we let $T^{tt}=\delta(z)$ (as we are confined to the sheet $z=0$). As the LHS $\propto \delta(z)$, RHS $\propto \delta(z)$.

As a first derivative cannot be a delta-function (proof not shown), and as $\Lambda=0$, we see
\begin{equation}
\Phi'' \propto \delta(z)
\end{equation}

We also have
\begin{align}
8\pi T^{xx}&=0=\Phi'^2+\Phi'\Psi'+\Psi'^2+\Phi'' + \Psi'' + \Lambda \label{xx 1}\\
8\pi T^{zz}&=0=\Psi'(2\Phi' + \Psi') + \Lambda \label{zz 1}
\end{align}
Subtracting~(\ref{zz 1}) from~(\ref{xx 1}) we find
\begin{equation}
\Phi'' = -\Psi''
\end{equation}
which implies that $\Phi''$ must be a delta-function.

Similarly, subtracting~(\ref{xx 1}) from~(\ref{zz 1}) we find
\begin{align}
0&=(\Phi')^2-\Phi' \Psi'\\
\therefore \Psi'&=0 \text{ or } \Phi'=\Psi'
\end{align}

However, both of these possibilities contradict with our previous findings! Hence we cannot generate~(\ref{eq3}) with a sheet of cold matter.


\begin{framed}
\textbf{ii.} By thinking about a suitable initial value problem qualitatively, speculate why this seemingly natural system - a static dust sheet - is impossible to assemble ``from the ground up'' in general relativity, even though there is no obstacle in a Newtonian context. What might ``go wrong''? Do not try calculating anything, unless you are in the mood to win a Nobel Prize.
\end{framed}

For each individual particle we add, a gravitational or electric force would act on the other particles currently in the ``sheet'' causing a distortion to the structure. This means that a flat sheet structure would be impossible to form!


\begin{framed}
\textbf{(g)} Consider the special case $\Psi=0$, i.e. warped time, Euclidean space, no preferential
warping of space in the $z$ direction relative to the $x$ and $y$ directions.
\end{framed}

\begin{framed}
\textbf{i.} In the bulk ($z=0$), where there is zero stress-energy, show that $\Lambda=0$ must
hold to obtain a self-consistent solution of the form~(\ref{eq3}).
\end{framed}

With zero stress-energy, we have
\begin{equation}
8\pi T^{tt}=0=\Psi'(2\Phi' + \Psi) + \Lambda = \Lambda
\end{equation}
since $\Psi=0\Rightarrow \Psi'=0$. Hence we see
\begin{equation}
\Lambda = 0
\end{equation}
as required.


\begin{framed}
\textbf{ii.} Hence show that the stress-energy must vanish on the sheet $z = 0$ too.
\end{framed}

At $z=0$, there will be no mass as $\Lambda=0,\Psi=0$. Hence no invariant mass $E_{m} \Rightarrow$ no energy density $\Rightarrow T^{tt}=0$.


\begin{framed}
\textbf{iii.} Prove that~(\ref{eq3}) reduces to the Rindler metric for a uniformly accelerated frame.
What is the implied acceleration?

In the special case $\Phi = 0$, a similar bulk-then-sheet approach yields the Minkowski
metric.
\end{framed}

We start with

\begin{equation}
ds^2=-e^{2\Phi(z)}dt^2 + dx^2 + dy^2 + dz^2
\end{equation}

For small $z$, we can Taylor expand, so that we have
\begin{equation}
ds^2 = -\left(1+2\Phi'(z)\right)dt^2 + dx^2 + dy^2 + dz^2
\end{equation}

This is of the form of an accelerating frame with acceleration $2\Phi'(z)$!

\begin{framed}
\textbf{(h)} Consider the special case $\Psi=\Phi$. This corresponds to preferentially warping space in $z$ relative to $x$ and $y$, i.e. the ``symmetric'' coordinates $(t, x, y)$ are treated as a Minkowski subspace with warping in the $z$ coordinate. The system is the four dimensional analogue of the five-dimensional \emph{Randall-Sundrum 1-brane} spacetime, which revolutionized the study of brane worlds 
\end{framed}


\begin{framed}
\textbf{i.} In the bulk, show that one obtains
\begin{equation}
\Phi=g |z|,\label{eq12}
\end{equation}
if $\Phi(z)$ vanishes at $z = 0$ without loss of generality. In~(\ref{eq12}), $g$ is an integration
constant with units of acceleration.
\end{framed}


\begin{align}
\Phi = \Psi \Rightarrow 0 &= 3(\Phi') + \Lambda\\
\Rightarrow\frac{d\Phi}{dz}&= \pm \sqrt{-\frac{\Lambda}{3}}\\
\Rightarrow \Phi &= \pm \sqrt{-\frac{\Lambda}{3}}|z|,\quad\text{as z is symmetric}\\
&=g|z|
\end{align}

as required.

\begin{framed}
\textbf{ii.} Prove that $g$ satisfies the fine-tuning condition
\begin{equation}
\Lambda=-3g^2.\label{eq13}
\end{equation}
Comment on the physical significance of the sign of $\Lambda$.
\end{framed}

\begin{equation}
g=\pm \sqrt{-\frac{\Lambda}{3}} \therefore \Lambda = -3g^2
\end{equation}
Positive $\Lambda$ gives an imaginary $g$, which is not physical. $\Rightarrow \Lambda$ is negative. Physically, this means that space contracts!

\begin{framed}
\textbf{iii.} Assume that the stress-energy is confined to the sheet, i.e. $T^{\mu\nu}\propto \delta(z)$. Prove that the stress-energy tensor takes the physically unusual form
\begin{equation}
T^{\mu\nu}=(2\pi)^{-1}g\delta(z)~\diag(-1,1,1,0).\label{eq14}
\end{equation}
Equation~(\ref{eq14}) implies fine tuning between $T^{\mu\nu}$ and $\Lambda$ via $g$.
\end{framed}

We have

\begin{align}
\Phi = g|z| \Rightarrow \Phi''&= 2g\delta(z)
\intertext{since}
\frac{d\Phi}{dz}&=\text{sgn}(z)\\
\frac{d}{dz}\text{sgn}(z)&=z\delta(z)
\end{align}

We also find that
\begin{align}
3(\Phi')^2&=-\Lambda\\
\Rightarrow 8\pi T^{tt}&=-e^{2g|z|}(4\Phi'')\\
\Rightarrow T{tt}&= -e^{2g|z|} (2\pi)^{-1} \delta(z)\\
8\pi T^{xx}&= e^{2g|z|}(4\Phi'')\\
T^{xx}&=(2\pi)^{-1}e^{2g|z|} \delta(z)\\
T^{yy}&=T^{xx}\\
8\pi T^{zz}&=0\\
\Rightarrow T^{zz}&=0
\end{align}
which leaves us with
\begin{equation}
T^{\mu\nu}=(2\pi)^{-1}g~\delta(z)\diag(-1,1,1,0),
\end{equation}
as required.


\begin{framed}
\textbf{iv.} Interpret physically the two cases $g < 0$ and $g > 0$.
\end{framed}

In the case of $g<0$, there is positive energy density, so the normal stress (pressure) is negative.

In the case of $g>0$, there is negative energy density, so there normal stress (pressure) is positive.


\begin{framed}
\textbf{v.} Interpret physically the signs of $T^{tt}$ and $T^{xx} = T^{yy}$ for $g > 0$.
\end{framed}

For $g>0$, $T^{tt}$ is negative, $T^{xx}=T^{yy}$ is positive.

\begin{framed}
\textbf{vi.}  Explain physically why $T^{zz} = 0$ makes sense in terms of a plausible microscopic
(``particulate'') model of the sheet.
\end{framed}

$T^{zz}$ is pressure in the $z-$direction. We note that pressure leads to motion in $z$ of the sheet; this motion would cause the sheet to buckle. However, our sheet is flat which implies there must not be any pressure in $z \Rightarrow T^{zz}=0$.

\begin{framed}
\textbf{vii.} Prove that it is impossible to build the sheet out of a perfect fluid or dust.
\end{framed}

For a perfect fluid the stress-energy tensor is given by (see \emph{Schutz} (4.36))
\begin{equation}
T^{\mu\nu}=\text{diag}(\rho,p,p,p)\label{perfect fluid}
\end{equation}

However, our $T^{\mu\nu}$ is confined to the sheet,
\begin{equation}
T^{\mu\nu}\propto \text{diag}(-1,1,1,0)\label{sheet}
\end{equation}

In the perfect fluid~(\ref{perfect fluid}), we see the last two elements of the tensor are identical. However, on the sheet~(\ref{sheet}) we see the last two elements are different $(1\neq 0) \Rightarrow $ we cannot build a sheet out of a perfect fluid!

\begin{framed}
\textbf{(i)} A tennis player standing on the sheet hits a tennis ball into the $z>0$ half-space with initial 4-velocity $u(0)$.
\end{framed}

\begin{framed}
\textbf{i.} Starting from the geodesic equations in the form
\begin{equation}
\frac{du_{\nu}}{d\tau}=\frac{1}{2}g_{\alpha\beta,\nu}u^{\alpha}u^{\beta},\label{eq15}
\end{equation}
identify three constants of the motion. In~(\ref{eq15}), $\tau$ denotes the proper time as measured in the ball's momentarily comoving reference frame.
\end{framed}

The geodesic equation is given by
\begin{equation}
\frac{du_{\nu}}{d\tau}=\frac{1}{2}g_{\alpha\beta,\nu}u^{\alpha}u^{\beta}
\end{equation}
where we have the metric $g_{\alpha\beta}=\diag(-e^{2gz},e^{2gz},e^{2gz},1)$. The metric is dependent only for $z$, so we find
\begin{equation}
\frac{du_{\nu}}{d\tau}=0\quad\text{for $\nu=t,x,y$}\\
\end{equation}

Hence $u_{t},u_{x},u_{y}$ are constants of the motion.


\begin{framed}
\textbf{ii.} From the z component of~(\ref{eq15}), show that
\begin{equation}
\frac{du^z}{d\tau}=-g[1+(u^z)^2].\label{eq16}
\end{equation}
What is unusual physically about this acceleration?
\end{framed}

Starting from the geodesic equation we have
\begin{align}
\frac{du_{z}}{d\tau}&=ge^{2gz}\left(-(u^t)^2+(u^x)^2+(u^y)^2\right)
\Rightarrow \frac{du^{z}}{d\tau}=ge^{2gz}\left(-(u^t)^2+(u^x)^2+(u^y)^2\right)
\end{align}
where we raise the index ($u_{z} \to u^{z}$) by multiplying both sides by $g^{zz}=1$.

As $\vec{u}\cdot\vec{u}=-1$, we find
\begin{align}
e^{2gz}\left(-(u^t)^2+(u^x)^2+(u^y)^2\right)&=-1\\
\Rightarrow e^{2gz}\left(-(u^t)^2+(u^x)^2+(u^y)^2\right)&=-\left(1+(u^z)^2\right)
\end{align}

Hence we see
\begin{equation}
\frac{du^z}{d\tau}=-g\left[1+(u^z)\right]^2
\end{equation}
as required!

We note that the acceleration in $z$ depends on the velocity of the ball. In our own experience if we throw a ball up, we would expect the the acceleration on the ball to be a constant due to gravity. In this case, however, we see that this acceleration is dependent on the velocity! This is quite unsual.

\begin{framed}
\textbf{iii.} Similarly or otherwise, prove the equivalent result
\begin{equation}
\left(\frac{dz}{d\tau}\right)^2=-1+e^{-2gz}\{1+[u^z(0)]^2\}.\label{eq17}
\end{equation}
\end{framed}

We start with
\begin{equation}
\vec{u}\cdot\vec{u}=-1 \Rightarrow e^{-2gz}\left(-(u_t)^2+(u_x)^2+(u_y)^2\right)
+(u_z)^2=-1
\end{equation}

Using the initial condition $z=0$ at $\tau=0$, and the fact that $u_x,u_y,u_t$ are constants of the motion:
\begin{align}
-\left(u_t(0)\right)^2+\left(u_x(0)\right)^2+\left(u_y(0)\right)^2+\left(u_z(0)\right)^2&=-1 \\
\Rightarrow e^{4gz}\left[\left(u^t(0)\right)^2+\left(u^x(0)\right)^2+\left(u^y\right)^2\right]
&=-\left[1+\left(u^z(0)\right)^2\right]\\
\Rightarrow e^{2gz}\left[\left(u^t\right)^2+\left(u^x\right)^2+\left(u^y\right)^2\right]
&=-e^{-2gz}\left[1+\left(u^z(0)\right)^2\right]\label{LHS of eqn, 1}
\end{align}
as $u^{t},u^{x},u^{y}$ are constants of motion (e.g. $u^t = u^t(0)$).

Also, we note
\begin{equation}
\vec{u}\cdot \vec{u}=-1 \Rightarrow e^{2gz}\left(-(u^t)^2+(u^x)^2+(u^y)^2\right)=
-\left(1+(u^z)^2\right)\label{LHS of eqn, 2}
\end{equation}

Equating the LHS's of~(\ref{LHS of eqn, 1}) and~(\ref{LHS of eqn, 2}) we find
\begin{align}
-e^{-2gz}\left[1+\left(u^z(0)\right)^2\right] & = -\left(1+(u^z)^2\right)\\
\Rightarrow \left(\frac{dz}{d\tau}\right)^2&=-1+e^{-2gz}\left[1+\left(u^z(0)\right)^2\right]
\end{align}

as required.

\begin{framed}
\textbf{iv.} Solve either~(\ref{eq16}) or~(\ref{eq17}) to obtain
\begin{equation}
z=g^{-1}\ln[\cos g\tau+u^z(0)\sin g\tau].\label{eq18}
\end{equation}
Please note that there are many valid ways to prove~(\ref{eq16})-(\ref{eq18}).
\end{framed}


We recall from~(\ref{eq17}) that
\begin{align}
\frac{du^z}{d\tau}&=-g\left(1+(u^z)^2\right)\\
\Rightarrow \int\frac{du^z}{\left(1+(u^z)^2\right)}&=\int -g~d\tau\\
\tan^{-1}(u^z)&=-g\tau+c_{1}
\intertext{where $c_1$ is some constant.}
u^{z}&=\tan(-g\tau + c_1)
\intertext{We can use the initial condition $z=0,\tau=0$ to determine the value of the constant}
u^{z}(0)&=\tan(c)\\
\Rightarrow c&=\tan^{-1}\left(u^z(0)\right)
\intertext{We shall substitute this value in later (leaving it as $c_1$ for now, for convenience.}
u^{z}&=\frac{dz}{dt}\\
\Rightarrow \int dz&=\int d\tau~\tan(-g\tau+c_1)
\end{align}

Evaluating this integral, we can now calculate $z$.

\begin{align}
z&=\frac{1}{g}\log\left[\cos(c_1-g\tau)\right]+c_2
\intertext{where $c_2$ is some constant,}
&=\frac{1}{g}\log\left[\cos\Big(\tan^{-1}\big(u^z(0)-g\tau\big)\Big)\right]\\
&=\frac{1}{g}\log\left[\cos\Big(\tan^{-1}\big(u^z(0)\big)\Big)\cos(g\tau)
+ \sin\Big(\tan^{-1}\big(u^z(0)\big)\Big)\sin(g\tau)\right] + c_2\label{cosab}
\intertext{where we have used the identity $\cos(\alpha-\beta)=\cos\alpha \cos \beta
+ \sin\alpha \sin \beta$. We now also recall that}
\cos\left[\tan^{-1}(x)\right]&=\frac{1}{\sqrt{1+x^2}}\\
\sin\left[\tan^{-1}(x)\right]&=\frac{x}{\sqrt{1+x^2}}\\
\Rightarrow z&=\frac{1}{g}\log\left(\frac{\cos(g\tau)+u^z(0)\sin(g\tau)}
{\sqrt{1+(u^z(0))^2}}\right)+c_2\\
&=\frac{1}{g}\log\left[\cos(g\tau)+u^z(0)\sin(g\tau)\right]
-\frac{1}{g}\log\left[\sqrt{1+(u^z(0))^2}\right]+c_2
\intertext{We can determine $c_2$ using the initial conditions $z=0,\tau=0$,}
\Rightarrow 0&=\underbrace{\frac{1}{g}\log(1)}_{=0}+\frac{1}
{\log\left[\sqrt{1+(u^z(0))^2}\right]}+c_2\\
\Rightarrow c_2 &= \frac{1}{g}\log\left[\sqrt{1+(U^z(0))^2}\right]
\intertext{Hence we conclude}
z&=g^{-1}\log[\cos g\tau+u^z(0)\sin g\tau].
\end{align}
as required.


\begin{framed}
\textbf{(j)} Show that the ball reaches the apex of its trajectory at proper time
\begin{equation}
\tau_{\text{top}}=g^{-1}\tan^{-1}u^z(0),
\end{equation}
and that the associated coordinate time is
\begin{equation}
t_{\text{top}}=\frac{u^z(0)\{1+[u^x(0)]^2+[u^z(0)]^2\}^{1/2}}{g\{1+[u^z(0)]^2\}},
\end{equation}
in the coordinates in which~(\ref{eq3}) is written. Interestingly, $\tau_{\text{top}}$ is finite, no matter
how hard the ball is struck initially. Does the same hold true for $t_{\text{top}}$?
\end{framed}

We know there will be a maximum (or minimum) in $z$ at $\frac{dz}{d\tau}=0$; hence we can solve for $\tau$ at this point by
\begin{align}
\frac{dz}{d\tau}&=\tan\left[-g\tau + \tan^{-1}(u^z(0))\right]\\
\Rightarrow 0 &= \tan\left[-g\tau_{\text{top}} + \tan^{-1}(u^z(0))\right]\\
g\tau_{\text{top}}&=\tan^{-1}\left[u^z(0)\right]\\
\Rightarrow \tau_{\text{top}}&=g^{-1}\tan^{-1}\left[u^z(0)\right]
\end{align}
as required. \footnote{We have not shown that $z$ takes a maximum (rather than a minimum) at this $\tau$, however it would not be be difficult to show this (e.g. by checking the sign of $\frac{d^2 z}{d\tau^2}$).}

We have
\begin{equation}
\frac{du_t}{d\tau}=0 \Rightarrow u_t = c
\end{equation}
where $c$ is some constant. By initial conditions $z=0,\tau=0$ we see
\begin{equation}
c=U_t(0)
\end{equation}

Starting from $u_t$, we can calculate
\begin{align}
u_t=g_{tt}u^t=-e^{2gz}u^t&=c\\
\Rightarrow u^t = -e^{-2gz}\cdot c\\
\Rightarrow \frac{dt}{dz}&=-u_t(0)e^{-2gz}\\
dt&=-d\tau\cdot u_t(0) e^{-2gz}\\
&=-d\tau\cdot u_t(0)e^{-2g\left\{g^{-1}\log\left[\cos(g\tau)+u^z(0)\sin(g\tau)\right]\right\}}\\
&=-d\tau\cdot u_t(0) e^{\log\left\{\left[\cos(g\tau)+u^z(0)\sin(g\tau)\right]^{-2}\right\}}\\
&=-d\tau\cdot u_t(0)\left[\cos(g\tau) + u^z(0)\sin(g\tau)\right]^{-2}\\
&=-d\tau \cdot u_t(0)\left(1+(u^z(0))^2\right)^{-1}\cos\left[-g\tau + \tan^{-1}(u^z(0))\right]^{-2}
\intertext{where we have performed the opposite steps from~(\ref{cosab}).}
\Rightarrow t&=g^{-1}u_t(0)\left(1+(U^z(0))^2\right)^{-1}
\tan\left(-g\tau + \tan^{-1}(u^z(0))\right) + d
\intertext{We can determine the value of our constant $d$ by considering the initial condition $t=0,\tau=0$}
\Rightarrow d&=-g^{-1}u_t(0)\left(1+(u^z(0))^2\right)^{-1}u^z(0)
\intertext{Since $t_{\text{top}}$ will occur when $\tau=\tau_{\text{top}}=g^{-1}\tan^{-1}(u^z(0)),$}
t_{\text{top}}&=-g^{-1} u_t(0)\left(1+(u^z(0))^2\right)^{-1}u^z(0)
\end{align}

Using $U_{t}(0)=u^t(0)$ and $-1=\vec{u}\cdot \vec{u}$, we can see
\begin{equation}
U^{t}(0)=\pm\sqrt{1+(u^x(0))^2+(u^y(0))^2+(u^z(0))^2}
\end{equation}

Taking the negative square root gives us the expression for $t_{\text{top}}$

\begin{equation}
t_{\text{top}}=\frac{u^z(0)\{1+[u^x(0)]^2+[u^z(0)]^2\}^{1/2}}{g\{1+[u^z(0)]^2\}},
\end{equation}

Since we can choose how fast we throw the ball, this can be infinite!

\begin{framed}
\textbf{(k)} Show that the horizontal distance traversed by the ball from its launch point to where it lands on the sheet, as measured in the coordinates associated with~(\ref{eq3}), is given by
\begin{equation}
x_{\text{land}}=\frac{2u^x(0)u^z(0)}{g\{1+[u^z(0)]^2\}}.\label{eq21}
\end{equation}
How does~(\ref{eq21}) differ qualitatively from Newtonian projectile motion? Should we expect $x_{\text{land}}$ to equal the $x$-component of the initial 3-velocity multiplied by $2t_{\text{top}}$,
since the metric does not depend on $x$?
\end{framed}

Starting from
\begin{equation}
\frac{du_x}{d\tau}=0
\end{equation}
we find that $u_x$ is constant; i.e.
\begin{equation}
 u_x=c
\end{equation}
From initial conditions, we have
\begin{equation}
c=u_x (0)
\end{equation}

We then calculate
\begin{align}
u_x =g_{xx}u^x &= e^{2gz}u^x\\
u^x &= e^{-2gz}u_x(0)\\
\frac{dx}{d\tau}&=u_x(0)\left[\cos(g\tau)+u^z(0)\sin(g\tau)\right]\\
\Rightarrow x&=  -g^{-1}u^x(0)\left[1+u^z(0)\right]\tan\left[-g\tau + \tan^{-1}u^z(0)\right] + d
\end{align}
similarly to what we have done previously (note, $d$ is some constant).

Our initial conditions of $x=0,\tau=0$ allow us to calculate $d$ as
\begin{equation}
d=g^{-1}U^x(0)\left[1+(u^z(0))^2\right]^{-1}u^z(0)
\end{equation}

We realise that $x_{\text{land}}$ will occur at $\tau_{\text{land}}$. We assume that $\tau_{\text{land}}=2\tau_{\text{top}}$, but in any case we shall prove this below.

\begin{align}
z&=g^{-1}\log\left[\cos(g\tau)+u^z(0)\sin(g\tau)\right]\\
\Rightarrow 0&=g^{-1}\log\left[\sqrt{1+(u^z(0))^2}\cos(-g\tau_{\text{land}}
+\tan^{-1}(u^z(0)))\right]\\
\Rightarrow \frac{1}{\sqrt{1+(u^z(0))^2}}&=\cos\left(-g\tau_{\text{land}}+\tan^{-1}
(u^z(0))\right)\\
\cos^{-1}\left(\frac{1}{\sqrt{1+(u^z(0))^2}}\right)&=-g\tau_{\text{top}}+\tan^{-1}(u^z(0))
\intertext{Using the relation $\cos^{-1}\left(\frac{1}{\sqrt{1+x^2}}\right)=\tan^{-1}(x)$, we have}
\tan^{-1}(u^z(0))&=-g\tau_{\text{land}}+\tan^{-1}\left(U^z(0)\right)\\
\Rightarrow \tau_{\text{land}}&=2g^{-1}\tan^{-1}\left(U^z(0)\right)
\intertext{which is as we expected,}
\tau_{\text{land}}&=2\tau_{\text{top}}
\end{align}

Using this value of $\tau$, we can determine $x_{\text{max}}$.

\begin{align}
x_{\text{max}}&=-g^{-1}u^x(0)\left(1+(u^z(0))^2\right)^{-1}
\tan\left[-2\tan^{-1}(u^z(0))+\tan^{-1}(u^z(0))\right] + c\\
&=g^{-1}u^x(0)\left[1+(u^z(0))^2\right]^{-1}u^z(0)+g^{-1}u^x(0)\left[1+(u^z(0))^2\right]^{-1}
u^z(0)\\
\Rightarrow x_{\text{land}}&=\frac{2u^x(0)u^z(0)}{g\{1+[u^z(0)]^2\}}\label{x top}
\end{align}
as required.

We should not expect $x_{\text{land}}=\text{initial x-velocity}\times 2t_{\text{top}}$; we see from~(\ref{x top}) that for large $u^z(0)$ we have
\begin{equation}
x_{\text{land}}\sim \frac{2u^x(0)}{g u^z(0)}
\end{equation}
Since we are free to choose $u^x(0)$ and $u^z(0)$ we see that $x_{\text{land}}$ can be whatever value we dictate (not just  $\text{initial x-velocity}\times t_{\text{land}}$!).



\begin{framed}
\textbf{(l)} Finally, suppose that instead of a tennis ball we launch a laser beam into the $z > 0$ half-space with initial 4-momentum $p(0)$ per photon.
\end{framed}

\begin{framed}
\textbf{i.} Let $\lambda$ be an affine parameter tracing out the light ray. Starting from the geodesic equations or otherwise, solve for the photon’s trajectory. You should find
\begin{equation}
z=g^{-1}\ln[1+p^z(0)g\lambda].
\end{equation}
In other words, the photon never falls back to the sheet, no matter how much stress-energy the sheet contains.
\end{framed}

Starting from the geodesic equations, we have that $p_x,p_y,p_t$ are constants of the motion. We also consider $\vec{p}\cdot \vec{p}$. That is,

\begin{equation}
e^{2gz}\left[-(p^t)^2+(p^x)^2+(p^y)^2\right]+(p^z)^2=0\label{li eq1}
\end{equation}

Now, using the initial condition $z=0,\tau=0$, we find

\begin{equation}
\left[-(p_t(0))^2+(p_x(0))^2+(p_y(0))^2\right]+(p_z(0))^2=0
\end{equation}

\begin{equation}
e^{-2gz}\left[p^z(0)\right]^2=-e^{2gz}\left[-(p^t)^2+(p^x)^2+(p^y)^2\right]\label{li eq2}
\end{equation}
where we use the inverse metric to raise indices. Combining equations~(\ref{li eq1}) and~(\ref{li eq2}) we find
\begin{align}
(p^z)^2&=e^{-2gz}\left[p^z(0)\right]^2\\
\Rightarrow \frac{dz}{d\lambda}&=e^{-gz}p^z(0)\\
\int dz~e^{gz}&=\int d\lambda~p^z(0)\\
g^{-1}e^{gz}&=p^z(0)\lambda+c\\
\Rightarrow e^{gz}&=g\lambda p^z(0)+c
\intertext{where $c$ is some constant. We can determine the value of $c$ by considering the initial conditions $z=0,\lambda=0$}
\Rightarrow c&=1\\
\Rightarrow e^{gz}&=g\lambda p^z(0) + 1\\
gz&=\log\left[g\lambda p^z(0)+1\right]\\
\Rightarrow z&= g^{-1}\log\left[1+g\lambda p^z(0)\right]
\end{align}

as required.





\begin{framed}
\textbf{ii.} Consider an intergalactic variant of the Pound-Rebka experiment, in which we fly horizontally (in the $x$ direction, say) in a rocket at constant speed $V$ [as measured in the coordinates associated with~(\ref{eq3})] while maintaining a constant altitude $z_{\text{rocket}}$ above the sheet. Show that the laser frequency measured by the stationary emitter $\nu_{\text{em}}$, and the frequency measured by the experimentalist in the rocket, $\nu_{\text{rec}}$, are related according to
\begin{equation}
\frac{\nu_{\text{rec}}}{\nu_{\text{em}}}=\frac{e^{-gz_{\text{rocket}}}}{(1-V^2)^{1/2}}
\left[1-\frac{Vp^x(0)}{\{[p^x(0)]^2+[p^z(0)]^2\}^{1/2}}\right].
\end{equation}

Most of the physics in this question applies in five dimensions too. It is interesting to think about it in the context of Randall-Sundrum brane worlds.
\end{framed}

For a photon, we have the energy $E=h\nu$, and its energy measured by an observer is given by $E=-\vec{p}\cdot \vec{u}$.

For the stationary emitter
\begin{align}
U&=(1,0,0,0)\\
\Rightarrow E&=p^{t}e^{2gz}\\
&=p^t(0)
\end{align}
as the emitter is at $z=0$ (and we recall that $p^t$ is a constant of the motion).

For the receiver, $\vec{u}=(Uut,Vu^t,0,0)$, as $\vec{u}\cdot\vec{u}=-1$.

\begin{align}
-1&=(u^t)^2(-1+V^2)e^{2gz}\\
\Rightarrow u^t&=\left[(1-V^2)^{-1}e^{-2gz}\right]^{1/2}\\
&=(1-V^2)^{-1/2} e^{-gz_{\text{rocket}}}
\end{align}

Therefore $-\vec{p}\cdot\vec{u}$ for the receiver is given by
\begin{equation}
-g_{\alpha\beta}p^{\alpha}u^{\beta}=u^{t}e^{2gz}p^t - u^t e^{2gz} V p^{x}
\end{equation}

So,
\begin{equation}
\frac{\nu_{\text{rec}}}{\nu_{\text{emit}}}=\frac{E_{\text{rec}}}{E_{\text{emit}}}
=\frac{u^t e^{2gz}(p^{t}-Vp^{x})}{p^{t}(0)}
=\frac{e^{-gz}e^{2gz}(p^{t}-Vp^{x}}{(1-V^2)^{1/2}p^{t}(0)}
\end{equation}

Also, we see
\begin{align}
p^t&=e^{-2gz}p_t=-e^{-2gt}p_t(0)=e^{-2gz}p^{t}(0)\\
p^x&=e^{-2gz}p_x=e^{-2gz}p_{x}(0)=e^{-2gz}p^x(0)
\end{align}

\begin{equation}
\frac{e^{gz}\left[e^{-2gz}p^t(0)-Ve^{-2gz}p^x(0)\right]}{(1-V^2)^{1/2}p^t(0)}
=\frac{e^{-gz}}{(1-V^2)^{1/2}}
\end{equation}

\begin{equation}
-(p^t(0))^2+(p^x(0))^2+(p^y(0))^2+(p^z(0))^2=0
\end{equation}
as $\vec{p}\cdot\vec{p}=0$, with $z=\tau=0$.

\begin{equation}
\Rightarrow p^t(0)=\sqrt{(p^x(0))^2+(p^y(0))^2+(p^z(0))^2}
\end{equation}

\begin{align}
\Rightarrow \frac{\nu_{\text{rec}}}{\nu_{\text{emit}}}&=
\frac{e^{-gz}}{(1-V^2)^{1/2}}\left(1-\frac{Vp^x(0)}
{\sqrt{(p^x(0))^2+(p^y(0))^2+(p^z(0))^2}}\right)\\
&=\frac{e^{-gz}}{(1-V^2)^{1/2}}\left(1-
\frac{Vp^{x}(0)}{\sqrt{(p^x(0))^2+(p^z(0))^2}}\right)
\end{align}
where because of symmetry, we have rescaled $p^{x'}(0)$ similar to $t_{\text{top}}$ in Question (j).

This is the result as required!


\end{document}







































